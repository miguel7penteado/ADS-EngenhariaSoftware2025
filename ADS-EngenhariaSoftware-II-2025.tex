% Options for packages loaded elsewhere
\PassOptionsToPackage{unicode}{hyperref}
\PassOptionsToPackage{hyphens}{url}
%
\documentclass[
]{book}
\usepackage{amsmath,amssymb}
\usepackage{iftex}
\ifPDFTeX
  \usepackage[T1]{fontenc}
  \usepackage[utf8]{inputenc}
  \usepackage{textcomp} % provide euro and other symbols
\else % if luatex or xetex
  \usepackage{unicode-math} % this also loads fontspec
  \defaultfontfeatures{Scale=MatchLowercase}
  \defaultfontfeatures[\rmfamily]{Ligatures=TeX,Scale=1}
\fi
\usepackage{lmodern}
\ifPDFTeX\else
  % xetex/luatex font selection
\fi
% Use upquote if available, for straight quotes in verbatim environments
\IfFileExists{upquote.sty}{\usepackage{upquote}}{}
\IfFileExists{microtype.sty}{% use microtype if available
  \usepackage[]{microtype}
  \UseMicrotypeSet[protrusion]{basicmath} % disable protrusion for tt fonts
}{}
\makeatletter
\@ifundefined{KOMAClassName}{% if non-KOMA class
  \IfFileExists{parskip.sty}{%
    \usepackage{parskip}
  }{% else
    \setlength{\parindent}{0pt}
    \setlength{\parskip}{6pt plus 2pt minus 1pt}}
}{% if KOMA class
  \KOMAoptions{parskip=half}}
\makeatother
\usepackage{xcolor}
\usepackage{color}
\usepackage{fancyvrb}
\newcommand{\VerbBar}{|}
\newcommand{\VERB}{\Verb[commandchars=\\\{\}]}
\DefineVerbatimEnvironment{Highlighting}{Verbatim}{commandchars=\\\{\}}
% Add ',fontsize=\small' for more characters per line
\usepackage{framed}
\definecolor{shadecolor}{RGB}{248,248,248}
\newenvironment{Shaded}{\begin{snugshade}}{\end{snugshade}}
\newcommand{\AlertTok}[1]{\textcolor[rgb]{0.94,0.16,0.16}{#1}}
\newcommand{\AnnotationTok}[1]{\textcolor[rgb]{0.56,0.35,0.01}{\textbf{\textit{#1}}}}
\newcommand{\AttributeTok}[1]{\textcolor[rgb]{0.13,0.29,0.53}{#1}}
\newcommand{\BaseNTok}[1]{\textcolor[rgb]{0.00,0.00,0.81}{#1}}
\newcommand{\BuiltInTok}[1]{#1}
\newcommand{\CharTok}[1]{\textcolor[rgb]{0.31,0.60,0.02}{#1}}
\newcommand{\CommentTok}[1]{\textcolor[rgb]{0.56,0.35,0.01}{\textit{#1}}}
\newcommand{\CommentVarTok}[1]{\textcolor[rgb]{0.56,0.35,0.01}{\textbf{\textit{#1}}}}
\newcommand{\ConstantTok}[1]{\textcolor[rgb]{0.56,0.35,0.01}{#1}}
\newcommand{\ControlFlowTok}[1]{\textcolor[rgb]{0.13,0.29,0.53}{\textbf{#1}}}
\newcommand{\DataTypeTok}[1]{\textcolor[rgb]{0.13,0.29,0.53}{#1}}
\newcommand{\DecValTok}[1]{\textcolor[rgb]{0.00,0.00,0.81}{#1}}
\newcommand{\DocumentationTok}[1]{\textcolor[rgb]{0.56,0.35,0.01}{\textbf{\textit{#1}}}}
\newcommand{\ErrorTok}[1]{\textcolor[rgb]{0.64,0.00,0.00}{\textbf{#1}}}
\newcommand{\ExtensionTok}[1]{#1}
\newcommand{\FloatTok}[1]{\textcolor[rgb]{0.00,0.00,0.81}{#1}}
\newcommand{\FunctionTok}[1]{\textcolor[rgb]{0.13,0.29,0.53}{\textbf{#1}}}
\newcommand{\ImportTok}[1]{#1}
\newcommand{\InformationTok}[1]{\textcolor[rgb]{0.56,0.35,0.01}{\textbf{\textit{#1}}}}
\newcommand{\KeywordTok}[1]{\textcolor[rgb]{0.13,0.29,0.53}{\textbf{#1}}}
\newcommand{\NormalTok}[1]{#1}
\newcommand{\OperatorTok}[1]{\textcolor[rgb]{0.81,0.36,0.00}{\textbf{#1}}}
\newcommand{\OtherTok}[1]{\textcolor[rgb]{0.56,0.35,0.01}{#1}}
\newcommand{\PreprocessorTok}[1]{\textcolor[rgb]{0.56,0.35,0.01}{\textit{#1}}}
\newcommand{\RegionMarkerTok}[1]{#1}
\newcommand{\SpecialCharTok}[1]{\textcolor[rgb]{0.81,0.36,0.00}{\textbf{#1}}}
\newcommand{\SpecialStringTok}[1]{\textcolor[rgb]{0.31,0.60,0.02}{#1}}
\newcommand{\StringTok}[1]{\textcolor[rgb]{0.31,0.60,0.02}{#1}}
\newcommand{\VariableTok}[1]{\textcolor[rgb]{0.00,0.00,0.00}{#1}}
\newcommand{\VerbatimStringTok}[1]{\textcolor[rgb]{0.31,0.60,0.02}{#1}}
\newcommand{\WarningTok}[1]{\textcolor[rgb]{0.56,0.35,0.01}{\textbf{\textit{#1}}}}
\usepackage{longtable,booktabs,array}
\usepackage{calc} % for calculating minipage widths
% Correct order of tables after \paragraph or \subparagraph
\usepackage{etoolbox}
\makeatletter
\patchcmd\longtable{\par}{\if@noskipsec\mbox{}\fi\par}{}{}
\makeatother
% Allow footnotes in longtable head/foot
\IfFileExists{footnotehyper.sty}{\usepackage{footnotehyper}}{\usepackage{footnote}}
\makesavenoteenv{longtable}
\usepackage{graphicx}
\makeatletter
\def\maxwidth{\ifdim\Gin@nat@width>\linewidth\linewidth\else\Gin@nat@width\fi}
\def\maxheight{\ifdim\Gin@nat@height>\textheight\textheight\else\Gin@nat@height\fi}
\makeatother
% Scale images if necessary, so that they will not overflow the page
% margins by default, and it is still possible to overwrite the defaults
% using explicit options in \includegraphics[width, height, ...]{}
\setkeys{Gin}{width=\maxwidth,height=\maxheight,keepaspectratio}
% Set default figure placement to htbp
\makeatletter
\def\fps@figure{htbp}
\makeatother
\setlength{\emergencystretch}{3em} % prevent overfull lines
\providecommand{\tightlist}{%
  \setlength{\itemsep}{0pt}\setlength{\parskip}{0pt}}
\setcounter{secnumdepth}{5}
\usepackage{booktabs}
\ifLuaTeX
  \usepackage{selnolig}  % disable illegal ligatures
\fi
\usepackage[]{natbib}
\bibliographystyle{plainnat}
\usepackage{bookmark}
\IfFileExists{xurl.sty}{\usepackage{xurl}}{} % add URL line breaks if available
\urlstyle{same}
\hypersetup{
  pdftitle={ADS - Engenharia de Software 2025 - Anotações de aula},
  pdfauthor={Professor Miguel Suez Xve Penteado},
  hidelinks,
  pdfcreator={LaTeX via pandoc}}

\title{ADS - Engenharia de Software 2025 - Anotações de aula}
\author{Professor Miguel Suez Xve Penteado}
\date{2025-03-16}

\begin{document}
\maketitle

{
\setcounter{tocdepth}{1}
\tableofcontents
}
\chapter*{Sobre estas anotações}\label{sobre-estas-anotauxe7uxf5es}
\addcontentsline{toc}{chapter}{Sobre estas anotações}

Estas anotações são apenas lembretes das aulas expostas em sala, durante a disciplina de ENGENHARIA DE SOFTWARE.

\section{ACESSO AO GITBOOK CELULAR}\label{acesso-ao-gitbook-celular}

\section{\texorpdfstring{\url{https://miguel7penteado.github.io/ADS-EngenhariaSoftware2025}}{https://miguel7penteado.github.io/ADS-EngenhariaSoftware2025}}\label{httpsmiguel7penteado.github.ioads-engenhariasoftware2025}

\includegraphics{images/clipboard-3692082511.png}

\section{APP EPUB ANDROID}\label{app-epub-android}

\section{\texorpdfstring{\textbf{Moon+ Reader}}{Moon+ Reader}}\label{moon-reader}

\includegraphics{images/qrcode/leitor_epub/MoonReaderPlus.jpg}

\chapter{Livros Texto da Disciplina}\label{livros-texto-da-disciplina}

\subsubsection{``Engenharia de Software'' do autor ``Roger S Pressman''}\label{engenharia-de-software-do-autor-roger-s-pressman}

\includegraphics{images/livros/engenharia_software_pressman.jpg}

\begin{longtable}[]{@{}
  >{\raggedright\arraybackslash}p{(\columnwidth - 2\tabcolsep) * \real{0.5000}}
  >{\raggedright\arraybackslash}p{(\columnwidth - 2\tabcolsep) * \real{0.5000}}@{}}
\toprule\noalign{}
\endhead
\bottomrule\noalign{}
\endlastfoot
\textbf{Autor(es)} & \href{https://www.indicalivros.com/autores?q=Roger\%20S.\%20Pressman}{Roger S. Pressman} \\
\textbf{Editora} & Pearson \\
\textbf{Idioma} & Português \\
\textbf{ISBN} & 8534602379 9788534602372 \\
\textbf{Formato} & Capa comum \\
\textbf{Páginas} & 1056 \\
\textbf{Código Biblioteca} & \\
\end{longtable}

\subsubsection{``Engenharia de Software'' do autor ``Ian Sommerville''}\label{engenharia-de-software-do-autor-ian-sommerville}

\includegraphics{images/livros/engenharia_software_sommerville.jpg}

\begin{longtable}[]{@{}ll@{}}
\toprule\noalign{}
\endhead
\bottomrule\noalign{}
\endlastfoot
\textbf{Autor(es)} & Ian SommerVille \\
\textbf{Editora} & Pearson \\
\textbf{Idioma} & Português \\
\textbf{ISBN} & 9788588639072 \\
\textbf{Formato} & Capa comum \\
\textbf{Páginas} & 768 \\
\textbf{Código Biblioteca} & \\
\end{longtable}

Calendário das aulas

\paragraph{FEVEREIRO 2025}\label{fevereiro-2025}

\begin{longtable}[]{@{}lll@{}}
\toprule\noalign{}
\endhead
\bottomrule\noalign{}
\endlastfoot
Data & Dia da semana & Aulas \\
4 de fevereiro & Terça-feira & Recesso \\
11 de fevereiro & Terça-feira & Recesso \\
18 de fevereiro & Terça-feira & Aula Inaugural \\
25 de fevereiro & Terça-feira & Qualidade de Software I \\
\end{longtable}

\paragraph{MARÇO 2025}\label{maruxe7o-2025}

\begin{longtable}[]{@{}lll@{}}
\toprule\noalign{}
\endhead
\bottomrule\noalign{}
\endlastfoot
Data & Dia da semana & Aulas \\
4 de março & Terça-feira & Carnaval \\
11 de março & Terça-feira & Verificação e Validação de Software I \\
18 de março & Terça-feira & Verificação e Validação de Software II \\
25 de março & Terça-feira & \\
\end{longtable}

\paragraph{ABRIL DE 2025}\label{abril-de-2025}

\begin{longtable}[]{@{}lll@{}}
\toprule\noalign{}
\endhead
\bottomrule\noalign{}
\endlastfoot
Data & Dia da semana & Aulas \\
1 de abril & Terça-feira & Prova NP1 \\
8 de abril & Terça-feira & Manutenção de software I \\
15 de abril & Terça-feira & Manutenção de software II \\
22 de abril & Terça-feira & Manutenção de software III \\
29 de abril & Terça-feira & Manutenção de software IV \\
\end{longtable}

\paragraph{MAIO DE 2025}\label{maio-de-2025}

\begin{longtable}[]{@{}lll@{}}
\toprule\noalign{}
\endhead
\bottomrule\noalign{}
\endlastfoot
Data & Dia da semana & Aulas \\
6 de maio & Terça-feira & Gerência de Configuração \\
13 de maio & Terça-feira & Revisão \\
20 de maio & Terça-feira & Prova NP2 \\
27 de maio & Terça-feira & Substitutiva \\
\end{longtable}

\paragraph{JUNHO DE 2025}\label{junho-de-2025}

\begin{longtable}[]{@{}lll@{}}
\toprule\noalign{}
\endhead
\bottomrule\noalign{}
\endlastfoot
Data & Dia da semana & Aulas \\
3 de junho & Terça-feira & Plantão \\
10 de junho & Terça-feira & Plantão \\
17 de junho & Terça-feira & Exame \\
24 de junho & Terça-feira & \\
\end{longtable}

\chapter{Alunos 2025}\label{alunos-2025}

\subsection{Turma DS2P40}\label{turma-ds2p40}

\begin{longtable}[]{@{}ll@{}}
\toprule\noalign{}
Matrícula & Nome do Aluno \\
\midrule\noalign{}
\endhead
\bottomrule\noalign{}
\endlastfoot
F35HFJ-1 & BEATRIZ ALMEIDA DA SILVA \\
R54885-6 & BRENO SOUZA MASCARENHAS \\
R19267-9 & CARLOS EDUARDO DA S GALDINO \\
R150FH-8 & DANILO LUCAS LOURENCO \\
G740IF-9 & GUSTAVO ALCANTARA NOBRE \\
G76IBD-7 & HELLEN REGINA B DOS SANTOS \\
F35EBD-4 & JOAO ALFREDO DA S BRENNER \\
R11835-5 & LUCAS ROSSE \\
G839GC-6 & PABLO HENRIQUE C ARAUJO \\
G61ICI-3 & THIAGO VERNIER LOUREIRO MAIA \\
\end{longtable}

\subsection{Turma DS3P40}\label{turma-ds3p40}

\begin{longtable}[]{@{}ll@{}}
\toprule\noalign{}
Matrícula & Nome do Aluno \\
\midrule\noalign{}
\endhead
\bottomrule\noalign{}
\endlastfoot
T736DG-3 & ANDRE LUIS RIGUEIRA ZANA \\
R06534-0 & BIANCA CAVALCANTE DOS SANTOS \\
G964AA-5 & CIBELE MARIA BARBOSA \\
R1007A-0 & DANIEL GOES CARVALHO SILVA \\
G98399-8 & DAVI PEREIRA DO VALE \\
F3567F-6 & EDUARDO MONTINO LACERDA \\
F35973-5 & FELIPE DE CAMPOS MOREIRA ALVES \\
R0622A-5 & FERNANDA VICTORIA D LO VACCO \\
R091EC-3 & GABRIEL ALVES BATISTA \\
G989DC-6 & GABRIEL PINHEIRO SOUZA \\
R08565-1 & GIOVANNY GOMES BRANDAO \\
R055AF-2 & GUILHERME NASCIMENTO R BARBOSA \\
N088EG-0 & GUILHERME R DE OLIVEIRA \\
R06229-5 & GUSTAVO SILVA DOS SANTOS \\
R07095-6 & HENRIQUE MOREIRA BOTELLA \\
R0823C-0 & HENRIQUE P DOS S FRANCISCO \\
G98BGB-2 & IGOR XAVIER DE MATTOS \\
G90JDE-2 & JOAO RICARDO DA SILVA JUNIOR \\
F3590G-2 & JOAO RICK GALDINO PEREIRA \\
R0567D-6 & JOAO VICTOR CARVALHO DE SOUZA \\
G9756E-6 & JOAO VICTOR DA SILVA MACHADO \\
G0249I-6 & JULIANA BORGES MOURA \\
F35937-9 & MATHEUS SERVULO CAJE \\
R10099-5 & MELISSA SCARPINATTI B DA SILVA \\
G8832G-1 & RENAN PRAZERES CLEMENTINO \\
F35CDF-2 & SERGIO ALEXANDRE A DO AMARAL \\
\end{longtable}

\subsection{Turma DS3Q40}\label{turma-ds3q40}

Com base nas informações da fonte \textbf{``DS3Q40.pdf''} e em nossa conversa anterior, apresento novamente a tabela com a coluna \textbf{Matrícula} (RA) e \textbf{Nome do Aluno} da turma \textbf{DS3Q40}:

\begin{longtable}[]{@{}ll@{}}
\toprule\noalign{}
Matrícula & Nome do Aluno \\
\midrule\noalign{}
\endhead
\bottomrule\noalign{}
\endlastfoot
G003II-9 & ALEX LIMA SILVA \\
G0327I-4 & AMANDA SIMONETTO DIAS \\
G02JDI-5 & ATILA WILLIAM F DE BARROS \\
R096DH-9 & BRENDA RUOTTI \\
R0087I-2 & GUSTAVO SILVA DE ARAUJO \\
G99JAH-4 & JESSICA SANTOS ANJOS \\
G8811G-1 & KAIKY ALVES MONTEIRO \\
G99319-5 & KLEBER WENDEL DE ALMEIDA RIBAS \\
G90EJA-1 & LEONARDO OLIVEIRA DOS SANTOS \\
G99ACJ-8 & LUCAS SILVA PINTO DE ASSIS \\
G99843-0 & MATHEUS ALVES LIMA \\
G996FJ-4 & MATHEUS DE OLIVEIRA MONTEIRO \\
G99JFJ-7 & MATHEUS RIBEIRO DE CAMPOS \\
G9931A-5 & PEDRO HENRIQUE CAMPOS LEAL \\
G012IF-3 & PEDRO PAULO VITALINO \\
R094GC-7 & RENAN DOS SANTOS FERREIRA \\
G96JFG-6 & RICHARD TRISTAN P GARCIA \\
G92GHH-8 & RODRIGO SANTOS ARAUJO \\
G977HG-0 & SIDNEI SERRAO DA SILVA \\
G003IC-0 & THIAGO DA SILVA SEIXEIRO \\
G99566-0 & YASMIN HELENA DE OLIVEIRA FERN \\
\end{longtable}

\begin{Shaded}
\begin{Highlighting}[]
\NormalTok{bookdown}\SpecialCharTok{::}\FunctionTok{render\_book}\NormalTok{()}
\end{Highlighting}
\end{Shaded}

\chapter*{INTRODUÇÃO A DISCIPLINA DE ENGENHARIA DE SOFTWARE}\label{introduuxe7uxe3o-a-disciplina-de-engenharia-de-software}
\addcontentsline{toc}{chapter}{INTRODUÇÃO A DISCIPLINA DE ENGENHARIA DE SOFTWARE}

Do que trata esta disciplina e o que quer dizer o termo que dá nome a ela ?

\section{O que é ENGENHARIA DE SOFTWARE}\label{o-que-uxe9-engenharia-de-software}

\includegraphics[width=0.95833in,height=\textheight]{images/pressman.jpg}

\begin{quote}
\textbf{Engenharia de Software}~\emph{é o processo de desenvolvimento de programas de computador, estruturas de dados e documentos.} (\textbf{\emph{Roger S. Pressman}})
\end{quote}

\includegraphics[width=1.0625in,height=\textheight]{images/sommerville.jpg}

\begin{quote}
\textbf{Engenharia de Software} \emph{é uma disciplina de engenharia que se preocupa com todo o processo de produção de software. Isso inclui desde a especificação do sistema até a sua manutenção.} (\textbf{Ian Sommerville})
\end{quote}

É atribuído a \textbf{Margaret Hamilton,} desenvolvedora do programa de navegação da APOLLO 11 a criação do termo ENGENHARIA DE SOFTWARE.

\includegraphics[width=0.5\textwidth,height=\textheight]{images/Margaret_Hamilton.jpg}

\chapter{QUALIDADE DE SOFTWARE}\label{qualidade-de-software}

\section{COMPLIANCE}\label{compliance}

Para que uma organização consiga fechar contrados de venda ou fornecimento com outra organização, especialmente quando o valor do contrato de venda ou prestação é muito alto, há um processo de checagem de COMPLIANCE:

\includegraphics{images/qualidade-geral/compliance.png}

\section{QUALIDADE}\label{qualidade}

O que é Qualidade ? (Definição ISO 9000)

\begin{quote}
Qualidade é definida como o grau em que um conjunto de características inerentes de um objeto satisfaz requisitos onde: \textbf{Características inerentes} São propriedades que fazem parte do objeto, onde:

\begin{itemize}
\item
  \textbf{Requisitos}: São as necessidades ou expectativas declaradas, geralmente implícitas ou obrigatórias;
\item
  \textbf{objeto} pode ser representado por um produto, serviço, processo, organização, sistema ou pessoa;
\end{itemize}
\end{quote}

\includegraphics{images/qualidade-geral/Qualidade.png}

\subsection{QUALIDADE APLICADA A PRODUTO}\label{qualidade-aplicada-a-produto}

O CONTROLE DE QUALIDADE do PRODUTO concentra-se em aperfeiçoar:

\begin{itemize}
\item
  as \textbf{características} e
\item
  o \textbf{desempenho} do produto em si,
\end{itemize}

visando atender às necessidades e expectativas dos clientes.

\includegraphics{images/qualidade-geral/Qualidade_do_Produto.png}

\subsection{QUALIDADE APLICADA A PROCESSO}\label{qualidade-aplicada-a-processo}

O CONTROLE DE QUALIDADE DE PROCESSO concentra-se em aperfeiçoar

\begin{itemize}
\item
  as \textbf{atividades} e
\item
  melhor \textbf{aplicação dos recursos}
\end{itemize}

utilizados para criar o produto, visando garantir a consistência e a eficácia da produção.

\includegraphics{images/qualidade-geral/Qualidade_do_Processo.png}

\subsection{CASO MACDONALDS - Qualidade de Produto e Processo}\label{caso-macdonalds---qualidade-de-produto-e-processo}

O filme ``Fome de Poder'' (``The Founder'', no original) narra a história real da ascensão da rede McDonald's, desde sua origem como uma pequena hamburgueria na Califórnia até se tornar um império global do fast-food.

\begin{itemize}
\tightlist
\item
  Reconhecimento da \textbf{qualidade do produto} - hamburguers McDonalds
\end{itemize}

\includegraphics{images/mac-donalds/cozinha-mac.jpg}

Reconhecimento da \textbf{Qualidade do Processo} de fabricação do Produto

\includegraphics{images/mac-donalds/modelo-mac.jpg}

\includegraphics{images/qualidade-geral/MacDonalds1.png}

\begin{itemize}
\tightlist
\item
  Reconhecimento da Capacidade de Franquia (Replicação):
\end{itemize}

\includegraphics[width=4.23958in,height=\textheight]{images/mac-donalds/Franquias.jpg}

\subsection{QUALIDADE NAS ORGANIZAÇÕES}\label{qualidade-nas-organizauxe7uxf5es}

\subsection{Família ISO 9000}\label{famuxedlia-iso-9000}

\textbf{A NBR ISO 9000} é um conjunto de normas técnicas que estabelecem diretrizes e padrões para a criação de um \textbf{Sistema de Gestão da Qualidade (SGQ)}.

O sistema SGQ (um si que pode ou não ser um pacote de software) deve mapear

\begin{longtable}[]{@{}
  >{\centering\arraybackslash}p{(\columnwidth - 8\tabcolsep) * \real{0.2000}}
  >{\centering\arraybackslash}p{(\columnwidth - 8\tabcolsep) * \real{0.2000}}
  >{\centering\arraybackslash}p{(\columnwidth - 8\tabcolsep) * \real{0.2000}}
  >{\centering\arraybackslash}p{(\columnwidth - 8\tabcolsep) * \real{0.2000}}
  >{\centering\arraybackslash}p{(\columnwidth - 8\tabcolsep) * \real{0.2000}}@{}}
\toprule\noalign{}
\endhead
\bottomrule\noalign{}
\endlastfoot
Áreas mapeadas por um sistema SGQ & PROCESSOS & POLÍTICAS & PROCEDIMENTOS & RESPONSABILIDADES \\
\end{longtable}

\includegraphics[width=5.71875in,height=\textheight]{images/qualidade-geral/ISO-9000-SGQ.png}

\subsection{Família ISO 14000}\label{famuxedlia-iso-14000}

\textbf{A NBR ISO 14000} é um conjunto de normas técnicas que tratam de GESTÃO AMBIENTAL nas organizações. Estabelecem normas e diretrizes para criar \textbf{(SI) Sistemas de Gestão Ambiental (SGA)}:

\includegraphics{images/qualidade-geral/ISO-14000-SGQ.png}

\subsection{Família ISO 27000}\label{famuxedlia-iso-27000}

\textbf{NBR ISO 27000}, trata de normas para \textbf{gestão segurança da Informação.} Fornecem um framework para a gestão da segurança da informação em organizações.

Especifica os requisitos para um para a criação de um(SI) Sistema de Gestão de Segurança da Informação (SGSI).

\includegraphics{images/qualidade-geral/ISO-27000-SGSI.png}

\subsection{Segmentos das Organizações e Adoção das normas de Qualidade}\label{segmentos-das-organizauxe7uxf5es-e-adouxe7uxe3o-das-normas-de-qualidade}

\includegraphics{images/qualidade-geral/segmentos.jpg}

\subsection{QUALIDADE NA ENGENHARIA DE SOFTWARE}\label{qualidade-na-engenharia-de-software}

A qualidade de software não define S.I.s

\subsection{Família NBR ISO 9126}\label{famuxedlia-nbr-iso-9126}

Focava na qualidade do produto de software, definindo um conjunto de parâmetros para padronizar a avaliação dessa qualidade. Ela se enquadrava no modelo de qualidade das normas da família 9000.

\includegraphics{images/qualidade-geral/ISO-9126-SGSI.png}

\subsection{Família NBR ISO 12207}\label{famuxedlia-nbr-iso-12207}

A norma ISO 12207 define um conjunto de processos para o ciclo de vida do software. Seu principal foco é estabelecer um framework padronizado para o desenvolvimento, manutenção e descarte de software, visando garantir a qualidade e a eficiência desses processos.

\includegraphics{images/modelos_processos_software/Cascata.jpg}

\includegraphics{images/qualidade-geral/ISO-12207-processos.png}

\subsection{Família NBR ISO 25000}\label{famuxedlia-nbr-iso-25000}

\textbf{A NBR ISO 25000}, também conhecida como SQuaRE (Software Product Quality Requirements and Evaluation - Requisitos e Avaliação da Qualidade de Produtos de Software), é uma série de normas internacionais que fornecem um subconjunto de normas para a avaliação da qualidade de produtos de software. Este subconjunto é formado pelas normas \textbf{ISO/IEC 25000} , \textbf{ISO/IEC 25010}, \textbf{ISO/IEC 25020}, \textbf{ISO/IEC 25030} e \textbf{ISO/IEC 25040}.

\includegraphics{images/qualidade-geral/ISO-25000-SQuaRE.png}

\chapter{Verificação de Validação de Software}\label{verificauxe7uxe3o-de-validauxe7uxe3o-de-software}

\includegraphics{images/clipboard-2445384906.png}

\section{Verificação de Softare:}\label{verificauxe7uxe3o-de-softare}

\begin{quote}
\textbf{Definição de Verificação de Software}: \emph{Assegurar que o software implementa corretamente uma função específica. ``Estamos criando o produto corretamente?}''.
\end{quote}

\section{Validação de Softare:}\label{validauxe7uxe3o-de-softare}

\begin{quote}
\textbf{Definição de Validação de Software}: \emph{Assegurar que o software foi criado e pode ser rastreado segundo os requisitos do cliente. ``Estamos criando o produto certo?''. Validação tem sucesso quando o software funciona de uma maneira que pode ser razoavelmente esperada pelo cliente.}
\end{quote}

\section{Classificação das Técnicas:}\label{classificauxe7uxe3o-das-tuxe9cnicas}

\subsection{Técnicas Estáticas}\label{tuxe9cnicas-estuxe1ticas}

As Técnicas Estáticas são Inspeções e revisões que analisam os requisitos do sistema, modelos de projeto e o código-fonte do programa sem executá-lo.

\subsection{Técnicas Dinâmicas}\label{tuxe9cnicas-dinuxe2micas}

As Técnicas Dinâmicas são testes de software, nos quais o sistema é executado com dados de testes simulados.

\subsection{Abordagens Formais}\label{abordagens-formais}

Já as abordagens formais são técnicas usadas para softwares críticos (usinas nucleares, navegação aérea, cirurgia robótica). Os processo de Prova de correção, o processo sala limpa (clean room).

\section{Revisões Técnicas: PASSEIO e INSPEÇÃO}\label{revisuxf5es-tuxe9cnicas-passeio-e-inspeuxe7uxe3o}

As \textbf{Revisões Técnicas (RT)} são Reuniões conduzidas por membros da equipe de software para avaliar a qualidade do software.

\subsubsection{As revisões técnicas podem ser ``Informais'' ou ``Formais''}\label{as-revisuxf5es-tuxe9cnicas-podem-ser-informais-ou-formais}

As \textbf{Revisões Informais} incluem testes de mesa e reuniões informais com colegas.

As \textbf{Revisões Técnicas Formais} são feitas com Reuniões estilizadas com papéis definidos, planejamento antecipado e manutenção de registros.

\subsection{Passeio (Walkthrough)}\label{passeio-walkthrough}

O produtor ``repassa'' o artefato, explicando o material, enquanto os revisores levantam questões com base em sua preparação prévia.

\subsection{Inspeção do produto}\label{inspeuxe7uxe3o-do-produto}

Uma pequena equipe verifica o código sistematicamente, procurando por possíveis erros e omissões.

\chapter{Estudo da Entrega \#01 - ERP Agrotec - Módulo Cadastros}\label{estudo-da-entrega-01---erp-agrotec---muxf3dulo-cadastros}

\section{Interface JanelaPrincipal}\label{interface-janelaprincipal}

Os aquivos estão na pasta ``ProjetoERP-AGROTEC\textbackslash01ModuloCadastros\textbackslash03codificacao\source'' do repositório da disciplina

Arquivo \emph{ERPAgroTech.py}

\begin{Shaded}
\begin{Highlighting}[]

\CommentTok{\# {-}*{-} coding: utf{-}8 {-}*{-}}

\CommentTok{\#\#\#\#\#\#\#\#\#\#\#\#\#\#\#\#\#\#\#\#\#\#\#\#\#\#\#\#\#\#\#\#\#\#\#\#\#\#\#\#\#\#\#\#\#\#\#\#\#\#\#\#\#\#\#\#\#\#\#\#\#\#\#\#\#\#\#\#\#\#\#\#\#\#\#}
\CommentTok{\#\# Python code generated with wxFormBuilder (version 4.2.1{-}0{-}g80c4cb6)}
\CommentTok{\#\# http://www.wxformbuilder.org/}
\CommentTok{\#\#}
\CommentTok{\#\# PLEASE DO *NOT* EDIT THIS FILE!}
\CommentTok{\#\#\#\#\#\#\#\#\#\#\#\#\#\#\#\#\#\#\#\#\#\#\#\#\#\#\#\#\#\#\#\#\#\#\#\#\#\#\#\#\#\#\#\#\#\#\#\#\#\#\#\#\#\#\#\#\#\#\#\#\#\#\#\#\#\#\#\#\#\#\#\#\#\#\#}

\ImportTok{import}\NormalTok{ wx}
\ImportTok{import}\NormalTok{ wx.xrc}

\ImportTok{import}\NormalTok{ gettext}
\NormalTok{\_ }\OperatorTok{=}\NormalTok{ gettext.gettext}


\ImportTok{from}\NormalTok{ CadastroClientes }\ImportTok{import}\NormalTok{ TipoCadastroClientes}

\CommentTok{\#\#\#\#\#\#\#\#\#\#\#\#\#\#\#\#\#\#\#\#\#\#\#\#\#\#\#\#\#\#\#\#\#\#\#\#\#\#\#\#\#\#\#\#\#\#\#\#\#\#\#\#\#\#\#\#\#\#\#\#\#\#\#\#\#\#\#\#\#\#\#\#\#\#\#}
\CommentTok{\#\# Class TipoJanelaPrincipal}
\CommentTok{\#\#\#\#\#\#\#\#\#\#\#\#\#\#\#\#\#\#\#\#\#\#\#\#\#\#\#\#\#\#\#\#\#\#\#\#\#\#\#\#\#\#\#\#\#\#\#\#\#\#\#\#\#\#\#\#\#\#\#\#\#\#\#\#\#\#\#\#\#\#\#\#\#\#\#}

\KeywordTok{class}\NormalTok{ TipoJanelaPrincipal ( wx.Frame ):}

   \KeywordTok{def} \FunctionTok{\_\_init\_\_}\NormalTok{( }\VariableTok{self}\NormalTok{, parent ):}
\NormalTok{       wx.Frame.}\FunctionTok{\_\_init\_\_}\NormalTok{ ( }\VariableTok{self}\NormalTok{, parent, }\BuiltInTok{id} \OperatorTok{=}\NormalTok{ wx.ID\_ANY, title }\OperatorTok{=}\NormalTok{ \_(}\StringTok{u"ERP AGROTEC {-} Entrega 01"}\NormalTok{), pos }\OperatorTok{=}\NormalTok{ wx.DefaultPosition, size }\OperatorTok{=}\NormalTok{ wx.Size( }\DecValTok{641}\NormalTok{,}\DecValTok{514}\NormalTok{ ), style }\OperatorTok{=}\NormalTok{ wx.DEFAULT\_FRAME\_STYLE}\OperatorTok{|}\NormalTok{wx.TAB\_TRAVERSAL )}

       \VariableTok{self}\NormalTok{.SetSizeHints( wx.DefaultSize, wx.DefaultSize )}

       \VariableTok{self}\NormalTok{.TipoMenuPrincipal }\OperatorTok{=}\NormalTok{ wx.MenuBar( }\DecValTok{0}\NormalTok{ )}
       \VariableTok{self}\NormalTok{.TipoMenuArquivo }\OperatorTok{=}\NormalTok{ wx.Menu()}
       \VariableTok{self}\NormalTok{.TipoMenuItemSair }\OperatorTok{=}\NormalTok{ wx.MenuItem( }\VariableTok{self}\NormalTok{.TipoMenuArquivo, wx.ID\_ANY, \_(}\StringTok{u"Sair"}\NormalTok{), wx.EmptyString, wx.ITEM\_NORMAL )}
       \VariableTok{self}\NormalTok{.TipoMenuArquivo.Append( }\VariableTok{self}\NormalTok{.TipoMenuItemSair )}

       \VariableTok{self}\NormalTok{.TipoMenuPrincipal.Append( }\VariableTok{self}\NormalTok{.TipoMenuArquivo, \_(}\StringTok{u"Arquivo"}\NormalTok{) )}

       \VariableTok{self}\NormalTok{.TipoMenuCadastro }\OperatorTok{=}\NormalTok{ wx.Menu()}
       \VariableTok{self}\NormalTok{.TipoMenuItemClientes }\OperatorTok{=}\NormalTok{ wx.MenuItem( }\VariableTok{self}\NormalTok{.TipoMenuCadastro, wx.ID\_ANY, \_(}\StringTok{u"Clientes"}\NormalTok{), wx.EmptyString, wx.ITEM\_NORMAL )}
       \VariableTok{self}\NormalTok{.TipoMenuCadastro.Append( }\VariableTok{self}\NormalTok{.TipoMenuItemClientes )}

       \VariableTok{self}\NormalTok{.TipoMenuItemFornecedores }\OperatorTok{=}\NormalTok{ wx.MenuItem( }\VariableTok{self}\NormalTok{.TipoMenuCadastro, wx.ID\_ANY, \_(}\StringTok{u"Fornecedores"}\NormalTok{), wx.EmptyString, wx.ITEM\_NORMAL )}
       \VariableTok{self}\NormalTok{.TipoMenuCadastro.Append( }\VariableTok{self}\NormalTok{.TipoMenuItemFornecedores )}

       \VariableTok{self}\NormalTok{.TipoMenuItemProdutos }\OperatorTok{=}\NormalTok{ wx.MenuItem( }\VariableTok{self}\NormalTok{.TipoMenuCadastro, wx.ID\_ANY, \_(}\StringTok{u"Produtos"}\NormalTok{), wx.EmptyString, wx.ITEM\_NORMAL )}
       \VariableTok{self}\NormalTok{.TipoMenuCadastro.Append( }\VariableTok{self}\NormalTok{.TipoMenuItemProdutos )}

       \VariableTok{self}\NormalTok{.TipoSubmenuRelatorios }\OperatorTok{=}\NormalTok{ wx.Menu()}
       \VariableTok{self}\NormalTok{.TipoMenuItemRelatorioClientes }\OperatorTok{=}\NormalTok{ wx.MenuItem( }\VariableTok{self}\NormalTok{.TipoSubmenuRelatorios, wx.ID\_ANY, \_(}\StringTok{u"Relatório de Clientes"}\NormalTok{), wx.EmptyString, wx.ITEM\_NORMAL )}
       \VariableTok{self}\NormalTok{.TipoSubmenuRelatorios.Append( }\VariableTok{self}\NormalTok{.TipoMenuItemRelatorioClientes )}

       \VariableTok{self}\NormalTok{.TipoMenuItemRelatorioFornecedores }\OperatorTok{=}\NormalTok{ wx.MenuItem( }\VariableTok{self}\NormalTok{.TipoSubmenuRelatorios, wx.ID\_ANY, \_(}\StringTok{u"Relatório de Fornecedores"}\NormalTok{), wx.EmptyString, wx.ITEM\_NORMAL )}
       \VariableTok{self}\NormalTok{.TipoSubmenuRelatorios.Append( }\VariableTok{self}\NormalTok{.TipoMenuItemRelatorioFornecedores )}

       \VariableTok{self}\NormalTok{.TipoMenuItemRelatorioProdutos }\OperatorTok{=}\NormalTok{ wx.MenuItem( }\VariableTok{self}\NormalTok{.TipoSubmenuRelatorios, wx.ID\_ANY, \_(}\StringTok{u"Relatório de Produtos"}\NormalTok{), wx.EmptyString, wx.ITEM\_NORMAL )}
       \VariableTok{self}\NormalTok{.TipoSubmenuRelatorios.Append( }\VariableTok{self}\NormalTok{.TipoMenuItemRelatorioProdutos )}

       \VariableTok{self}\NormalTok{.TipoMenuCadastro.AppendSubMenu( }\VariableTok{self}\NormalTok{.TipoSubmenuRelatorios, \_(}\StringTok{u"Relatórios"}\NormalTok{) )}

       \VariableTok{self}\NormalTok{.TipoMenuPrincipal.Append( }\VariableTok{self}\NormalTok{.TipoMenuCadastro, \_(}\StringTok{u"Cadastro"}\NormalTok{) )}

       \VariableTok{self}\NormalTok{.SetMenuBar( }\VariableTok{self}\NormalTok{.TipoMenuPrincipal )}

       \VariableTok{self}\NormalTok{.TipoBarraStatus }\OperatorTok{=} \VariableTok{self}\NormalTok{.CreateStatusBar( }\DecValTok{1}\NormalTok{, wx.STB\_SIZEGRIP, wx.ID\_ANY )}
       \VariableTok{self}\NormalTok{.TipoBarraStatus.SetBackgroundColour( wx.SystemSettings.GetColour( wx.SYS\_COLOUR\_INFOBK ) )}


       \VariableTok{self}\NormalTok{.Centre( wx.BOTH )}

       \CommentTok{\# Connect Events}
       \VariableTok{self}\NormalTok{.Bind( wx.EVT\_MENU, }\VariableTok{self}\NormalTok{.EventoTerminarPrograma,                }\BuiltInTok{id} \OperatorTok{=} \VariableTok{self}\NormalTok{.TipoMenuItemSair.GetId()                  )}
       \VariableTok{self}\NormalTok{.Bind( wx.EVT\_MENU, }\VariableTok{self}\NormalTok{.EventoAbrePainelClientes,              }\BuiltInTok{id} \OperatorTok{=} \VariableTok{self}\NormalTok{.TipoMenuItemClientes.GetId()              )}
       \VariableTok{self}\NormalTok{.Bind( wx.EVT\_MENU, }\VariableTok{self}\NormalTok{.EventoAbrePainelFornecedores,          }\BuiltInTok{id} \OperatorTok{=} \VariableTok{self}\NormalTok{.TipoMenuItemFornecedores.GetId()          )}
       \VariableTok{self}\NormalTok{.Bind( wx.EVT\_MENU, }\VariableTok{self}\NormalTok{.EventoAbrePainelProdutos,              }\BuiltInTok{id} \OperatorTok{=} \VariableTok{self}\NormalTok{.TipoMenuItemProdutos.GetId()              )}
       \VariableTok{self}\NormalTok{.Bind( wx.EVT\_MENU, }\VariableTok{self}\NormalTok{.EventoAbrePainelRelatorioClientes,     }\BuiltInTok{id} \OperatorTok{=} \VariableTok{self}\NormalTok{.TipoMenuItemRelatorioClientes.GetId()     )}
       \VariableTok{self}\NormalTok{.Bind( wx.EVT\_MENU, }\VariableTok{self}\NormalTok{.EventoAbrePainelRelatorioFornecedores, }\BuiltInTok{id} \OperatorTok{=} \VariableTok{self}\NormalTok{.TipoMenuItemRelatorioFornecedores.GetId() )}
       \VariableTok{self}\NormalTok{.Bind( wx.EVT\_MENU, }\VariableTok{self}\NormalTok{.EventoAbrePainelRelatorioProdutos,     }\BuiltInTok{id} \OperatorTok{=} \VariableTok{self}\NormalTok{.TipoMenuItemRelatorioProdutos.GetId()     )}

   \KeywordTok{def} \FunctionTok{\_\_del\_\_}\NormalTok{( }\VariableTok{self}\NormalTok{ ):}
       \ControlFlowTok{pass}

   \KeywordTok{def}\NormalTok{ MakeModal(}\VariableTok{self}\NormalTok{, modal}\OperatorTok{=}\VariableTok{True}\NormalTok{):}
       \ControlFlowTok{if}\NormalTok{ modal }\KeywordTok{and} \KeywordTok{not} \BuiltInTok{hasattr}\NormalTok{(}\VariableTok{self}\NormalTok{, }\StringTok{\textquotesingle{}\_disabler\textquotesingle{}}\NormalTok{):}
           \VariableTok{self}\NormalTok{.\_disabler }\OperatorTok{=}\NormalTok{ wx.WindowDisabler(}\VariableTok{self}\NormalTok{)}
       \ControlFlowTok{if} \KeywordTok{not}\NormalTok{ modal }\KeywordTok{and} \BuiltInTok{hasattr}\NormalTok{(}\VariableTok{self}\NormalTok{, }\StringTok{\textquotesingle{}\_disabler\textquotesingle{}}\NormalTok{):}
           \KeywordTok{del} \VariableTok{self}\NormalTok{.\_disabler}

   \CommentTok{\# Virtual event handlers, override them in your derived class}
   \KeywordTok{def}\NormalTok{ EventoTerminarPrograma( }\VariableTok{self}\NormalTok{, event ):}
\NormalTok{       event.Skip()}

   \KeywordTok{def}\NormalTok{ EventoAbrePainelClientes( }\VariableTok{self}\NormalTok{, event ):}
\NormalTok{       janelaClientes }\OperatorTok{=}\NormalTok{ TipoCadastroClientes(}\VariableTok{None}\NormalTok{)}
\NormalTok{       janelaClientes.MakeModal()}
\NormalTok{       janelaClientes.Show()}
       

   \KeywordTok{def}\NormalTok{ EventoAbrePainelFornecedores( }\VariableTok{self}\NormalTok{, event ):}
\NormalTok{       event.Skip()}

   \KeywordTok{def}\NormalTok{ EventoAbrePainelProdutos( }\VariableTok{self}\NormalTok{, event ):}
\NormalTok{       event.Skip()}

   \KeywordTok{def}\NormalTok{ EventoAbrePainelRelatorioClientes( }\VariableTok{self}\NormalTok{, event ):}
\NormalTok{       event.Skip()}

   \KeywordTok{def}\NormalTok{ EventoAbrePainelRelatorioFornecedores( }\VariableTok{self}\NormalTok{, event ):}
\NormalTok{       event.Skip()}

   \KeywordTok{def}\NormalTok{ EventoAbrePainelRelatorioProdutos( }\VariableTok{self}\NormalTok{, event ):}
\NormalTok{       event.Skip()}
\end{Highlighting}
\end{Shaded}

Arquivo main.py

\begin{Shaded}
\begin{Highlighting}[]

\CommentTok{\# {-}*{-} coding: utf{-}8 {-}*{-}}


\ImportTok{import}\NormalTok{ wx}

\ImportTok{from}\NormalTok{ ERPAgroTech }\ImportTok{import}\NormalTok{ TipoJanelaPrincipal}

\KeywordTok{class}\NormalTok{ Programa(TipoJanelaPrincipal):}
    \KeywordTok{def} \FunctionTok{\_\_init\_\_}\NormalTok{(}\VariableTok{self}\NormalTok{, parent):}
\NormalTok{        TipoJanelaPrincipal.}\FunctionTok{\_\_init\_\_}\NormalTok{(}\VariableTok{self}\NormalTok{, parent)}




\NormalTok{app }\OperatorTok{=}\NormalTok{ wx.App(}\VariableTok{False}\NormalTok{)    }\CommentTok{\# cria uma nova aplicação e não redireciona stdout e stderr para janela principal}
\NormalTok{frame }\OperatorTok{=}\NormalTok{ Programa(}\VariableTok{None}\NormalTok{) }\CommentTok{\# frame é uma janela de nível de topo}
\NormalTok{frame.MakeModal()}
\NormalTok{frame.Show()           }\CommentTok{\# Mostra a janela}
\NormalTok{app.MainLoop()         }\CommentTok{\# aplicação entra em loop até finalizar}
\end{Highlighting}
\end{Shaded}

\subsection{Como executar a janela principal}\label{como-executar-a-janela-principal}

\begin{enumerate}
\def\labelenumi{\arabic{enumi}.}
\item
  Baixar o e instalar o Python (preferencialmente a \href{https://www.python.org/ftp/python/3.9.0/python-3.9.0-amd64.exe}{versão 3.9 para Windows 10 ou 11})
\item
  Abrir uma janela do MS-DOS (prompt de comando) e mandar o utilitário \textbf{PIP} instalar o pacote \textbf{wxpython:}
\end{enumerate}

\begin{Shaded}
\begin{Highlighting}[]
\NormalTok{pip install {-}}\AttributeTok{{-}upgrade}\NormalTok{ wxphython}
\end{Highlighting}
\end{Shaded}

\begin{enumerate}
\def\labelenumi{\arabic{enumi}.}
\setcounter{enumi}{2}
\tightlist
\item
  Abrir uma janela do MS-DOS (prompt de comando) e mandar o utilitário e baixar o repositório da disciplina com a ferramenta GIT:
\end{enumerate}

\begin{Shaded}
\begin{Highlighting}[]
\NormalTok{git clone git@github.com:miguel7penteado}\AttributeTok{/ADS{-}EngenhariaSoftware2025}\NormalTok{.git}
\end{Highlighting}
\end{Shaded}

\begin{enumerate}
\def\labelenumi{\arabic{enumi}.}
\setcounter{enumi}{3}
\tightlist
\item
  Pelo MS-DOS entrar na pasta ProjetoERP-AGROTEC\textbackslash01ModuloCadastros\textbackslash03codificacao\source :
\end{enumerate}

\begin{Shaded}
\begin{Highlighting}[]
\BuiltInTok{cd}\NormalTok{ ADS{-}EngenhariaSoftware2025\textbackslash{}ProjetoERP{-}AGROTEC\textbackslash{}01ModuloCadastros\textbackslash{}03codificacao\textbackslash{}source}
\end{Highlighting}
\end{Shaded}

\begin{enumerate}
\def\labelenumi{\arabic{enumi}.}
\setcounter{enumi}{4}
\tightlist
\item
  Pelo MS-DOS mandar o interpretador python executar o ERP AGROTEC
\end{enumerate}

\begin{Shaded}
\begin{Highlighting}[]
\NormalTok{python3 main.py}
\end{Highlighting}
\end{Shaded}

\section{Cadastro de Clientes}\label{cadastro-de-clientes}

Acesso ao Banco de Dados na núvem POSTGRES para você testar o seu:

\begin{longtable}[]{@{}ll@{}}
\toprule\noalign{}
host: & pg-ads-engs2-miguel7penteado-ads-engs2.c.aivencloud.com \\
\midrule\noalign{}
\endhead
\bottomrule\noalign{}
\endlastfoot
porta: & 17135 \\
usuario: & SEU RA \\
senha: & SEU RA \\
banco: & banco-dados-ra \\
SSL: & require \\
\end{longtable}

OBS: substitua ``ra'' pelo seu ra, obviamente.

Cliente para testar via celular:

Android Postgresql Client

\url{https://play.google.com/store/apps/details?id=rafrobsystems.postgresclient&pcampaignid=web_share}

\includegraphics{images/clipboard-78059825.png}

\subsection{Tabela Clientes}\label{tabela-clientes}

\begin{Shaded}
\begin{Highlighting}[]
\KeywordTok{CREATE} \KeywordTok{TABLE}\NormalTok{ clientes}
\NormalTok{(}
\KeywordTok{id}         \DataTypeTok{varchar}\NormalTok{(}\DecValTok{15}\NormalTok{) }\KeywordTok{unique} \KeywordTok{not} \KeywordTok{null}\NormalTok{,}
\NormalTok{nome       }\DataTypeTok{varchar}\NormalTok{(}\DecValTok{500}\NormalTok{) }\KeywordTok{not} \KeywordTok{null}\NormalTok{ ,}
\NormalTok{endereco   }\DataTypeTok{varchar}\NormalTok{(}\DecValTok{500}\NormalTok{) }\KeywordTok{not} \KeywordTok{null}\NormalTok{ ,}
\NormalTok{nascimento }\DataTypeTok{date}
\NormalTok{);}
\end{Highlighting}
\end{Shaded}

\section{Cadastro de Fornecedores}\label{cadastro-de-fornecedores}

\section{Cadastro de Produtos}\label{cadastro-de-produtos}

\chapter{Introdução à Manutenção de Software}\label{introduuxe7uxe3o-uxe0-manutenuxe7uxe3o-de-software}

\section{1- Manutenção: definição e características''}\label{manutenuxe7uxe3o-definiuxe7uxe3o-e-caracteruxedsticas}

\subsection{Definição de Manutenção de Software:}\label{definiuxe7uxe3o-de-manutenuxe7uxe3o-de-software}

O processo de modificação de um produto de software após a entrega, para corrigir defeitos, melhorar o desempenho ou outros atributos {[}Sommerville{]}. A manutenção é uma parte importante da evolução do software {[}Sommerville{]}.

\subsection{Natureza da Mudança:}\label{natureza-da-mudanuxe7a}

Discutir por que o software precisa ser mantido, incluindo correções de bugs, adaptação a novos ambientes, implementação de novos requisitos e manutenção preventiva {[}Sommerville, Pressman{]}. A Primeira Lei da Engenharia de Sistemas afirma que não importa em qual estágio do ciclo de vida, o sistema mudará {[}Sommerville - referindo-se à inevitabilidade da mudança{]}.

\subsection{Tipos de Manutenção:}\label{tipos-de-manutenuxe7uxe3o}

Apresentar as categorias comuns de manutenção, como corretiva, adaptativa, perfectiva e preventiva {[}Pressman{]}. O livro ``ENGENHARIA DE SOFTWARE II.pdf'' lista ``Tipos de manutenção'' como um subtópico dentro de ``Manutenção de software''.

\subsection{Custos da Manutenção:}\label{custos-da-manutenuxe7uxe3o}

Mencionar que os custos de manutenção podem frequentemente exceder os custos iniciais de desenvolvimento {[}Sommerville, Pressman{]}.

\section{2- Introdução à Manutenibilidade}\label{introduuxe7uxe3o-uxe0-manutenibilidade}

\subsection{Definição Preliminar:}\label{definiuxe7uxe3o-preliminar}

Apresentar o conceito de manutenibilidade como a facilidade com que o software pode ser modificado {[}Sommerville, Pressman{]}. A manutenibilidade é um atributo essencial de um bom software {[}Sommerville{]} e um indicativo qualitativo da facilidade de corrigir, adaptar ou melhorar o software {[}Pressman, 74{]}.

\subsection{Importância da Manutenibilidade:}\label{importuxe2ncia-da-manutenibilidade}

Discutir por que a manutenibilidade é crucial para reduzir os custos e o esforço da manutenção a longo prazo {[}Sommerville, Pressman{]}. Qualidade e facilidade de manutenção são resultantes de um projeto bem feito {[}Pressman, 24{]}.

\chapter{APROFUNDANDO A MANUTENIBILIDADE E AS TÉCNICAS DE DESENVOLVIMENTO}\label{aprofundando-a-manutenibilidade-e-as-tuxe9cnicas-de-desenvolvimento}

\section{Manutenibilidade}\label{manutenibilidade}

\subsection{Atributos da Manutenibilidade:}\label{atributos-da-manutenibilidade}

Detalhar as características que influenciam a manutenibilidade, como modularidade, clareza do código, documentação adequada, complexidade, acoplamento e coesão {[}Pressman, Sommerville{]}. A boa prática de engenharia de software demanda modularidade efetiva, com alta coesão e baixo acoplamento {[}Pressman, 45{]}.

\subsection{Métricas de Manutenibilidade:}\label{muxe9tricas-de-manutenibilidade}

Introduzir a ideia de que a manutenibilidade pode ser avaliada e até mesmo medida através de métricas de software {[}Pressman, Sommerville{]}. O livro ``Engenharia-de-software-8-ed-roger-pressman.pdf'' menciona ``Métricas para manutenção'' no Capítulo 30.

\section{Técnicas de Desenvolvimento para a Manutenibilidade}\label{tuxe9cnicas-de-desenvolvimento-para-a-manutenibilidade}

\subsection{Princípios de Projeto:}\label{princuxedpios-de-projeto}

Discutir como princípios de projeto como separação de interesses, abstração, encapsulamento e baixo acoplamento impactam positivamente a manutenibilidade {[}Pressman, Sommerville{]}. A separação de interesses é um princípio-chave de projeto e implementação de software {[}Sommerville, 127{]}.

\subsection{Qualidade do Código:}\label{qualidade-do-cuxf3digo}

Enfatizar a importância de práticas de codificação limpa, convenções de estilo consistentes e evitar construções complexas para facilitar a compreensão e modificação do código.

\subsection{Documentação:}\label{documentauxe7uxe3o}

Ressaltar a necessidade de documentação interna (comentários no código) e externa (manuais, diagramas de projeto) para auxiliar na manutenção {[}Sommerville, Pressman{]}.

\chapter{PROCESSOS E PADRÕES NA MANUTENÇÃO DE SOFTWARE}\label{processos-e-padruxf5es-na-manutenuxe7uxe3o-de-software}

(Making off da aula )

\section{Processos de Manutenção}\label{processos-de-manutenuxe7uxe3o}

\subsection{Fluxo do Processo de Manutenção}\label{fluxo-do-processo-de-manutenuxe7uxe3o}

Apresentar as etapas típicas envolvidas em um processo de manutenção, como identificação da necessidade de mudança, análise da solicitação, projeto da modificação, implementação, teste e implantação {[}Pressman, Sommerville{]}.

\subsection{Gerenciamento de Mudanças}\label{gerenciamento-de-mudanuxe7as}

Discutir a importância de um processo formal de gerenciamento de mudanças para controlar as alterações aplicadas ao software durante a manutenção {[}Pressman, Sommerville{]}. O Princípio 6 da prática da engenharia de software é ``Gerencie mudanças'' {[}Pressman, 44{]}. O gerenciamento de configuração (abordado em nossa aula anterior) está intimamente ligado ao gerenciamento de mudanças na manutenção {[}Pressman, Sommerville{]}.

\section{Padrões de Desenvolvimento}\label{padruxf5es-de-desenvolvimento}

\subsection{Impacto na Manutenção}\label{impacto-na-manutenuxe7uxe3o}

Explicar como a adoção de padrões de desenvolvimento bem estabelecidos (arquiteturais, de projeto, de implementação) pode melhorar significativamente a manutenibilidade, promovendo consistência e compreensão do código {[}Pressman, Sommerville{]}. O projeto baseado em padrões é considerado {[}Pressman, 5, 13, 354{]}.

\subsection{Exemplos de Padrões Relevantes}\label{exemplos-de-padruxf5es-relevantes}

Apresentar exemplos de padrões que favorecem a manutenibilidade, como padrões de projeto GoF (Observer {[}Sommerville, 107{]}), padrões arquiteturais (camadas {[}Pressman, 15{]}), etc.

\section{Padrões de Manutenção}\label{padruxf5es-de-manutenuxe7uxe3o}

\subsection{Conceito e Exemplos}\label{conceito-e-exemplos}

Introduzir a ideia de padrões específicos para atividades de manutenção, como padrões de refatoração {[}Pressman, 12, 52{]} para melhorar a estrutura do código sem alterar seu comportamento externo.

\chapter{ABORDAGENS MODERNAS E ATIVIDADES DE APOIO À MANUTENÇÃO}\label{abordagens-modernas-e-atividades-de-apoio-uxe0-manutenuxe7uxe3o}

(Making Off)

\section{Desenvolvimento Baseado em Componentes e Impactos na Manutenção}\label{desenvolvimento-baseado-em-componentes-e-impactos-na-manutenuxe7uxe3o}

\subsection{Conceito de Desenvolvimento Baseado em Componentes (DBC):}\label{conceito-de-desenvolvimento-baseado-em-componentes-dbc}

Apresentar o DBC como uma abordagem que enfatiza a construção de sistemas a partir de componentes de software reutilizáveis {[}Pressman, Sommerville{]}. O desenvolvimento baseado em componentes é um modelo de processo especializado {[}Pressman, 9, 52{]}.

\subsection{Impactos na Manutenção:}\label{impactos-na-manutenuxe7uxe3o}

Discutir como o DBC pode influenciar a manutenção, facilitando a substituição, atualização ou reutilização de componentes, mas também introduzindo desafios relacionados à compatibilidade e dependências {[}Pressman{]}.

\section{Desenvolvimento Orientado a Aspectos e Impactos na Manutenção}\label{desenvolvimento-orientado-a-aspectos-e-impactos-na-manutenuxe7uxe3o}

\subsection{Conceito de Desenvolvimento Orientado a Aspectos (DOA):}\label{conceito-de-desenvolvimento-orientado-a-aspectos-doa}

Introduzir o DOA como uma técnica para modularizar interesses transversais (aspectos) que podem estar espalhados por vários módulos em um sistema orientado a objetos tradicional {[}Pressman, Sommerville{]}. O desenvolvimento de software orientado a aspectos é um modelo de processo especializado {[}Pressman, 9, 54{]}.

\subsection{Impactos na Manutenção:}\label{impactos-na-manutenuxe7uxe3o-1}

Explicar como o DOA pode melhorar a manutenibilidade ao isolar e gerenciar esses interesses transversais, tornando o código mais limpo e facilitando modificações em funcionalidades como logging, segurança ou tratamento de erros {[}Pressman, Sommerville{]}.

\section{Atividades de Apoio a Manutenção}\label{atividades-de-apoio-a-manutenuxe7uxe3o}

\subsection{Gerenciamento de Configuração na Manutenção:}\label{gerenciamento-de-configurauxe7uxe3o-na-manutenuxe7uxe3o}

Reforçar a importância do GC (que discutimos em uma aula anterior) para rastrear e controlar as mudanças realizadas durante a manutenção {[}Pressman, Sommerville{]}.

\subsection{Reengenharia:}\label{reengenharia}

Introduzir o conceito de reengenharia como uma forma de melhorar a manutenibilidade de sistemas legados através da reestruturação ou reimplementação {[}Pressman, Sommerville{]}. O Capítulo 36 do livro ``Engenharia-de-software-8-ed-roger-pressman.pdf'' trata de ``Manutenção e reengenharia''.

\subsection{Testes de Regressão:}\label{testes-de-regressuxe3o}

Destacar a necessidade de testes de regressão para garantir que as modificações realizadas durante a manutenção não introduzam novos defeitos {[}Pressman, Sommerville{]}.

\chapter{Gerência de Configuração}\label{geruxeancia-de-configurauxe7uxe3o}

(aula em processo de edição)

\includegraphics{images/clipboard-871701020.png}

\section{Introdução à Gerência de Configuração}\label{introduuxe7uxe3o-uxe0-geruxeancia-de-configurauxe7uxe3o}

\subsection{Definição de Gerência de Configuração (GC):}\label{definiuxe7uxe3o-de-geruxeancia-de-configurauxe7uxe3o-gc}

GC é o nome do processo geral de gerenciamento de um sistema de software em mudança. O objetivo do gerenciamento de configuração é apoiar o processo de integração do sistema para que todos os desenvolvedores possam acessar o código do projeto e os documentos relacionados de forma controlada, descobrir quais mudanças foram feitas, bem como compilar e ligar componentes para criar um sistema.

\subsection{A Natureza da Mudança em Software:}\label{a-natureza-da-mudanuxe7a-em-software}

A mudança é uma realidade para grandes sistemas. As necessidades e requisitos organizacionais se alteram durante a vida útil de um sistema, bugs precisam ser reparados e os sistemas necessitam se adaptar às mudanças em seu ambiente. De fato, a Primeira Lei da Engenharia de Sistemas afirma que não importa em qual estágio do ciclo de vida, o sistema mudará.

\subsection{Importância da GC:}\label{importuxe2ncia-da-gc}

Sem o gerenciamento de configuração, as mudanças aplicadas ao sistema podem ocorrer de forma descontrolada, levando a inconsistências, perda de trabalho e dificuldades na manutenção e evolução do software. A GC garante que as mudanças sejam aplicadas ao sistema de uma forma controlada.

\section{Elementos da Gerência de Configuração}\label{elementos-da-geruxeancia-de-configurauxe7uxe3o}

\subsection{Itens de Configuração de Software (ICIs)}\label{itens-de-configurauxe7uxe3o-de-software-icis}

Os itens que compõem todas as informações produzidas como parte do processo de software são chamados coletivamente de configuração de software. Isso inclui programas de computador (código fonte e executável), produtos que descrevem os programas (documentação para diversos stakeholders) e dados ou conteúdo. Um ICI é um elemento de informação com nome, que pode variar desde um simples diagrama UML até um documento de projeto completo. Diferentes versões de um ICI podem existir.

\subsection{Identificação:}\label{identificauxe7uxe3o}

Cada ICI deve ter um nome único para permitir seu rastreamento e gerenciamento.

\subsection{Controle de Versão:}\label{controle-de-versuxe3o}

Suporte para manter o controle das diferentes versões de ICIs ao longo do tempo. Isso permite rastrear o histórico de mudanças, reverter para versões anteriores e gerenciar múltiplas linhas de desenvolvimento.

\subsection{Controle de Mudanças:}\label{controle-de-mudanuxe7as}

O processo de garantia de que as mudanças em sistemas e componentes sejam registradas e mantidas para que as mudanças sejam gerenciadas e todas as versões de componentes sejam identificadas e armazenadas por todo o tempo de vida do sistema. Isso geralmente envolve solicitação de mudança, avaliação, aprovação e implementação controlada das alterações. Um formulário de solicitação de mudança pode ser utilizado.

\subsection{Auditoria de Configuração:}\label{auditoria-de-configurauxe7uxe3o}

Avaliações para garantir que os ICIs e seus registros correspondam à configuração real do software em um determinado momento.

\subsection{Relatório de Status:}\label{relatuxf3rio-de-status}

Documentação e comunicação sobre o status dos ICIs e das mudanças realizadas.

\section{Processo de Gerência de Configuração}\label{processo-de-geruxeancia-de-configurauxe7uxe3o}

\subsection{Planejamento da GC:}\label{planejamento-da-gc}

Definição das políticas, procedimentos e ferramentas a serem utilizadas para a GC.

\begin{longtable}[]{@{}
  >{\raggedright\arraybackslash}p{(\columnwidth - 2\tabcolsep) * \real{0.3087}}
  >{\raggedright\arraybackslash}p{(\columnwidth - 2\tabcolsep) * \real{0.6913}}@{}}
\toprule\noalign{}
\endhead
\bottomrule\noalign{}
\endlastfoot
\textbf{Etapa} & \textbf{Descrição} \\
\textbf{1-Identificação da Configuração} & Seleção dos itens de trabalho que serão controlados pela GC. \\
\textbf{2-Controle de Mudanças} & Implementação do processo para gerenciar solicitações de mudança, incluindo:

Solicitação formal de alteração.

Avaliação da alteração (impacto, custo, etc.).

Aprovação da alteração (geralmente por um Grupo de Controle de Alterações).

Implementação da alteração.

Verificação da alteração. \\
\textbf{3-Liberação da Configuração} & Preparação e disponibilização de versões específicas do software para teste, implantação ou entrega. \\
\textbf{4-Auditoria e Relatório da Configuração} & Verificação da conformidade com o plano de GC e comunicação do status da configuração. \\
\end{longtable}

\section{Ferramentas de Gerência de Configuração}\label{ferramentas-de-geruxeancia-de-configurauxe7uxe3o}

Muitas ferramentas de gerenciamento de configurações foram desenvolvidas para dar suporte aos processos de GC. Elas variam desde ferramentas simples que oferecem suporte a uma única tarefa (como rastreamento de bugs) até conjuntos complexos e caros de ferramentas integradas que oferecem suporte a todas as atividades de GC. Exemplos de funcionalidades comuns em ferramentas de GC incluem:

\begin{verbatim}
    *   Armazenamento e gerenciamento de versões de arquivos.
    *   Controle de acesso e permissões.
    *   Rastreamento de mudanças e histórico.
    *   Suporte a ramificações (branches) e merges.
    *   Gerenciamento de solicitações de mudança.
    *   Construção automatizada de sistemas.
*   Ambientes de Desenvolvimento Colaborativo (CDEs) como GForge, OneDesk e Rational Team Concert contêm recursos de gerenciamento de projeto e código, incluindo funcionalidades de GC.
\end{verbatim}

\section{GC em Contextos Ágeis e Tradicionais}\label{gc-em-contextos-uxe1geis-e-tradicionais}

A necessidade de gerenciamento de configuração é fundamental para todos os grandes sistemas desenvolvidos por equipes. Métodos ágeis também desenvolveram suas próprias terminologias de GC, às vezes para distinguir a abordagem ágil dos métodos tradicionais. Mesmo em desenvolvimento ágil, onde a mudança é bem-vinda, a GC é essencial para manter a organização e o controle sobre o software em evolução.

\section{Referências:}\label{referuxeancias}

Leitura dos Capítulos 25 de ``Engenharia-de-software-9-ed-Ian-Sommerville.pdf'' Leitura dos Capítulos 29 de ``Engenharia-de-software-8-ed-roger-pressman.pdf''. * Explorar ferramentas de Gerência de Configuração como Git, SVN, etc.

\chapter{Revisão para NP2}\label{revisuxe3o-para-np2}

\chapter{Revisão para a Substitutiva}\label{revisuxe3o-para-a-substitutiva}

\chapter{Apêndice I - Estudo da - ERP Agrotec}\label{apuxeandice-i---estudo-da---erp-agrotec}

\section{Entrega \#01 - Módulo Cadastros}\label{entrega-01---muxf3dulo-cadastros}

\section{Interface JanelaPrincipal}\label{interface-janelaprincipal-1}

Os aquivos estão na pasta ``ProjetoERP-AGROTEC\textbackslash01ModuloCadastros\textbackslash03codificacao\source'' do repositório da disciplina

Arquivo \emph{ERPAgroTech.py}

\begin{Shaded}
\begin{Highlighting}[]

\CommentTok{\# {-}*{-} coding: utf{-}8 {-}*{-}}

\CommentTok{\#\#\#\#\#\#\#\#\#\#\#\#\#\#\#\#\#\#\#\#\#\#\#\#\#\#\#\#\#\#\#\#\#\#\#\#\#\#\#\#\#\#\#\#\#\#\#\#\#\#\#\#\#\#\#\#\#\#\#\#\#\#\#\#\#\#\#\#\#\#\#\#\#\#\#}
\CommentTok{\#\# Python code generated with wxFormBuilder (version 4.2.1{-}0{-}g80c4cb6)}
\CommentTok{\#\# http://www.wxformbuilder.org/}
\CommentTok{\#\#}
\CommentTok{\#\# PLEASE DO *NOT* EDIT THIS FILE!}
\CommentTok{\#\#\#\#\#\#\#\#\#\#\#\#\#\#\#\#\#\#\#\#\#\#\#\#\#\#\#\#\#\#\#\#\#\#\#\#\#\#\#\#\#\#\#\#\#\#\#\#\#\#\#\#\#\#\#\#\#\#\#\#\#\#\#\#\#\#\#\#\#\#\#\#\#\#\#}

\ImportTok{import}\NormalTok{ wx}
\ImportTok{import}\NormalTok{ wx.xrc}

\ImportTok{import}\NormalTok{ gettext}
\NormalTok{\_ }\OperatorTok{=}\NormalTok{ gettext.gettext}


\ImportTok{from}\NormalTok{ CadastroClientes }\ImportTok{import}\NormalTok{ TipoCadastroClientes}

\CommentTok{\#\#\#\#\#\#\#\#\#\#\#\#\#\#\#\#\#\#\#\#\#\#\#\#\#\#\#\#\#\#\#\#\#\#\#\#\#\#\#\#\#\#\#\#\#\#\#\#\#\#\#\#\#\#\#\#\#\#\#\#\#\#\#\#\#\#\#\#\#\#\#\#\#\#\#}
\CommentTok{\#\# Class TipoJanelaPrincipal}
\CommentTok{\#\#\#\#\#\#\#\#\#\#\#\#\#\#\#\#\#\#\#\#\#\#\#\#\#\#\#\#\#\#\#\#\#\#\#\#\#\#\#\#\#\#\#\#\#\#\#\#\#\#\#\#\#\#\#\#\#\#\#\#\#\#\#\#\#\#\#\#\#\#\#\#\#\#\#}

\KeywordTok{class}\NormalTok{ TipoJanelaPrincipal ( wx.Frame ):}

   \KeywordTok{def} \FunctionTok{\_\_init\_\_}\NormalTok{( }\VariableTok{self}\NormalTok{, parent ):}
\NormalTok{       wx.Frame.}\FunctionTok{\_\_init\_\_}\NormalTok{ ( }\VariableTok{self}\NormalTok{, parent, }\BuiltInTok{id} \OperatorTok{=}\NormalTok{ wx.ID\_ANY, title }\OperatorTok{=}\NormalTok{ \_(}\StringTok{u"ERP AGROTEC {-} Entrega 01"}\NormalTok{), pos }\OperatorTok{=}\NormalTok{ wx.DefaultPosition, size }\OperatorTok{=}\NormalTok{ wx.Size( }\DecValTok{641}\NormalTok{,}\DecValTok{514}\NormalTok{ ), style }\OperatorTok{=}\NormalTok{ wx.DEFAULT\_FRAME\_STYLE}\OperatorTok{|}\NormalTok{wx.TAB\_TRAVERSAL )}

       \VariableTok{self}\NormalTok{.SetSizeHints( wx.DefaultSize, wx.DefaultSize )}

       \VariableTok{self}\NormalTok{.TipoMenuPrincipal }\OperatorTok{=}\NormalTok{ wx.MenuBar( }\DecValTok{0}\NormalTok{ )}
       \VariableTok{self}\NormalTok{.TipoMenuArquivo }\OperatorTok{=}\NormalTok{ wx.Menu()}
       \VariableTok{self}\NormalTok{.TipoMenuItemSair }\OperatorTok{=}\NormalTok{ wx.MenuItem( }\VariableTok{self}\NormalTok{.TipoMenuArquivo, wx.ID\_ANY, \_(}\StringTok{u"Sair"}\NormalTok{), wx.EmptyString, wx.ITEM\_NORMAL )}
       \VariableTok{self}\NormalTok{.TipoMenuArquivo.Append( }\VariableTok{self}\NormalTok{.TipoMenuItemSair )}

       \VariableTok{self}\NormalTok{.TipoMenuPrincipal.Append( }\VariableTok{self}\NormalTok{.TipoMenuArquivo, \_(}\StringTok{u"Arquivo"}\NormalTok{) )}

       \VariableTok{self}\NormalTok{.TipoMenuCadastro }\OperatorTok{=}\NormalTok{ wx.Menu()}
       \VariableTok{self}\NormalTok{.TipoMenuItemClientes }\OperatorTok{=}\NormalTok{ wx.MenuItem( }\VariableTok{self}\NormalTok{.TipoMenuCadastro, wx.ID\_ANY, \_(}\StringTok{u"Clientes"}\NormalTok{), wx.EmptyString, wx.ITEM\_NORMAL )}
       \VariableTok{self}\NormalTok{.TipoMenuCadastro.Append( }\VariableTok{self}\NormalTok{.TipoMenuItemClientes )}

       \VariableTok{self}\NormalTok{.TipoMenuItemFornecedores }\OperatorTok{=}\NormalTok{ wx.MenuItem( }\VariableTok{self}\NormalTok{.TipoMenuCadastro, wx.ID\_ANY, \_(}\StringTok{u"Fornecedores"}\NormalTok{), wx.EmptyString, wx.ITEM\_NORMAL )}
       \VariableTok{self}\NormalTok{.TipoMenuCadastro.Append( }\VariableTok{self}\NormalTok{.TipoMenuItemFornecedores )}

       \VariableTok{self}\NormalTok{.TipoMenuItemProdutos }\OperatorTok{=}\NormalTok{ wx.MenuItem( }\VariableTok{self}\NormalTok{.TipoMenuCadastro, wx.ID\_ANY, \_(}\StringTok{u"Produtos"}\NormalTok{), wx.EmptyString, wx.ITEM\_NORMAL )}
       \VariableTok{self}\NormalTok{.TipoMenuCadastro.Append( }\VariableTok{self}\NormalTok{.TipoMenuItemProdutos )}

       \VariableTok{self}\NormalTok{.TipoSubmenuRelatorios }\OperatorTok{=}\NormalTok{ wx.Menu()}
       \VariableTok{self}\NormalTok{.TipoMenuItemRelatorioClientes }\OperatorTok{=}\NormalTok{ wx.MenuItem( }\VariableTok{self}\NormalTok{.TipoSubmenuRelatorios, wx.ID\_ANY, \_(}\StringTok{u"Relatório de Clientes"}\NormalTok{), wx.EmptyString, wx.ITEM\_NORMAL )}
       \VariableTok{self}\NormalTok{.TipoSubmenuRelatorios.Append( }\VariableTok{self}\NormalTok{.TipoMenuItemRelatorioClientes )}

       \VariableTok{self}\NormalTok{.TipoMenuItemRelatorioFornecedores }\OperatorTok{=}\NormalTok{ wx.MenuItem( }\VariableTok{self}\NormalTok{.TipoSubmenuRelatorios, wx.ID\_ANY, \_(}\StringTok{u"Relatório de Fornecedores"}\NormalTok{), wx.EmptyString, wx.ITEM\_NORMAL )}
       \VariableTok{self}\NormalTok{.TipoSubmenuRelatorios.Append( }\VariableTok{self}\NormalTok{.TipoMenuItemRelatorioFornecedores )}

       \VariableTok{self}\NormalTok{.TipoMenuItemRelatorioProdutos }\OperatorTok{=}\NormalTok{ wx.MenuItem( }\VariableTok{self}\NormalTok{.TipoSubmenuRelatorios, wx.ID\_ANY, \_(}\StringTok{u"Relatório de Produtos"}\NormalTok{), wx.EmptyString, wx.ITEM\_NORMAL )}
       \VariableTok{self}\NormalTok{.TipoSubmenuRelatorios.Append( }\VariableTok{self}\NormalTok{.TipoMenuItemRelatorioProdutos )}

       \VariableTok{self}\NormalTok{.TipoMenuCadastro.AppendSubMenu( }\VariableTok{self}\NormalTok{.TipoSubmenuRelatorios, \_(}\StringTok{u"Relatórios"}\NormalTok{) )}

       \VariableTok{self}\NormalTok{.TipoMenuPrincipal.Append( }\VariableTok{self}\NormalTok{.TipoMenuCadastro, \_(}\StringTok{u"Cadastro"}\NormalTok{) )}

       \VariableTok{self}\NormalTok{.SetMenuBar( }\VariableTok{self}\NormalTok{.TipoMenuPrincipal )}

       \VariableTok{self}\NormalTok{.TipoBarraStatus }\OperatorTok{=} \VariableTok{self}\NormalTok{.CreateStatusBar( }\DecValTok{1}\NormalTok{, wx.STB\_SIZEGRIP, wx.ID\_ANY )}
       \VariableTok{self}\NormalTok{.TipoBarraStatus.SetBackgroundColour( wx.SystemSettings.GetColour( wx.SYS\_COLOUR\_INFOBK ) )}


       \VariableTok{self}\NormalTok{.Centre( wx.BOTH )}

       \CommentTok{\# Connect Events}
       \VariableTok{self}\NormalTok{.Bind( wx.EVT\_MENU, }\VariableTok{self}\NormalTok{.EventoTerminarPrograma,                }\BuiltInTok{id} \OperatorTok{=} \VariableTok{self}\NormalTok{.TipoMenuItemSair.GetId()                  )}
       \VariableTok{self}\NormalTok{.Bind( wx.EVT\_MENU, }\VariableTok{self}\NormalTok{.EventoAbrePainelClientes,              }\BuiltInTok{id} \OperatorTok{=} \VariableTok{self}\NormalTok{.TipoMenuItemClientes.GetId()              )}
       \VariableTok{self}\NormalTok{.Bind( wx.EVT\_MENU, }\VariableTok{self}\NormalTok{.EventoAbrePainelFornecedores,          }\BuiltInTok{id} \OperatorTok{=} \VariableTok{self}\NormalTok{.TipoMenuItemFornecedores.GetId()          )}
       \VariableTok{self}\NormalTok{.Bind( wx.EVT\_MENU, }\VariableTok{self}\NormalTok{.EventoAbrePainelProdutos,              }\BuiltInTok{id} \OperatorTok{=} \VariableTok{self}\NormalTok{.TipoMenuItemProdutos.GetId()              )}
       \VariableTok{self}\NormalTok{.Bind( wx.EVT\_MENU, }\VariableTok{self}\NormalTok{.EventoAbrePainelRelatorioClientes,     }\BuiltInTok{id} \OperatorTok{=} \VariableTok{self}\NormalTok{.TipoMenuItemRelatorioClientes.GetId()     )}
       \VariableTok{self}\NormalTok{.Bind( wx.EVT\_MENU, }\VariableTok{self}\NormalTok{.EventoAbrePainelRelatorioFornecedores, }\BuiltInTok{id} \OperatorTok{=} \VariableTok{self}\NormalTok{.TipoMenuItemRelatorioFornecedores.GetId() )}
       \VariableTok{self}\NormalTok{.Bind( wx.EVT\_MENU, }\VariableTok{self}\NormalTok{.EventoAbrePainelRelatorioProdutos,     }\BuiltInTok{id} \OperatorTok{=} \VariableTok{self}\NormalTok{.TipoMenuItemRelatorioProdutos.GetId()     )}

   \KeywordTok{def} \FunctionTok{\_\_del\_\_}\NormalTok{( }\VariableTok{self}\NormalTok{ ):}
       \ControlFlowTok{pass}

   \KeywordTok{def}\NormalTok{ MakeModal(}\VariableTok{self}\NormalTok{, modal}\OperatorTok{=}\VariableTok{True}\NormalTok{):}
       \ControlFlowTok{if}\NormalTok{ modal }\KeywordTok{and} \KeywordTok{not} \BuiltInTok{hasattr}\NormalTok{(}\VariableTok{self}\NormalTok{, }\StringTok{\textquotesingle{}\_disabler\textquotesingle{}}\NormalTok{):}
           \VariableTok{self}\NormalTok{.\_disabler }\OperatorTok{=}\NormalTok{ wx.WindowDisabler(}\VariableTok{self}\NormalTok{)}
       \ControlFlowTok{if} \KeywordTok{not}\NormalTok{ modal }\KeywordTok{and} \BuiltInTok{hasattr}\NormalTok{(}\VariableTok{self}\NormalTok{, }\StringTok{\textquotesingle{}\_disabler\textquotesingle{}}\NormalTok{):}
           \KeywordTok{del} \VariableTok{self}\NormalTok{.\_disabler}

   \CommentTok{\# Virtual event handlers, override them in your derived class}
   \KeywordTok{def}\NormalTok{ EventoTerminarPrograma( }\VariableTok{self}\NormalTok{, event ):}
\NormalTok{       event.Skip()}

   \KeywordTok{def}\NormalTok{ EventoAbrePainelClientes( }\VariableTok{self}\NormalTok{, event ):}
\NormalTok{       janelaClientes }\OperatorTok{=}\NormalTok{ TipoCadastroClientes(}\VariableTok{None}\NormalTok{)}
\NormalTok{       janelaClientes.MakeModal()}
\NormalTok{       janelaClientes.Show()}
       

   \KeywordTok{def}\NormalTok{ EventoAbrePainelFornecedores( }\VariableTok{self}\NormalTok{, event ):}
\NormalTok{       event.Skip()}

   \KeywordTok{def}\NormalTok{ EventoAbrePainelProdutos( }\VariableTok{self}\NormalTok{, event ):}
\NormalTok{       event.Skip()}

   \KeywordTok{def}\NormalTok{ EventoAbrePainelRelatorioClientes( }\VariableTok{self}\NormalTok{, event ):}
\NormalTok{       event.Skip()}

   \KeywordTok{def}\NormalTok{ EventoAbrePainelRelatorioFornecedores( }\VariableTok{self}\NormalTok{, event ):}
\NormalTok{       event.Skip()}

   \KeywordTok{def}\NormalTok{ EventoAbrePainelRelatorioProdutos( }\VariableTok{self}\NormalTok{, event ):}
\NormalTok{       event.Skip()}
\end{Highlighting}
\end{Shaded}

Arquivo main.py

\begin{Shaded}
\begin{Highlighting}[]

\CommentTok{\# {-}*{-} coding: utf{-}8 {-}*{-}}


\ImportTok{import}\NormalTok{ wx}

\ImportTok{from}\NormalTok{ ERPAgroTech }\ImportTok{import}\NormalTok{ TipoJanelaPrincipal}

\KeywordTok{class}\NormalTok{ Programa(TipoJanelaPrincipal):}
    \KeywordTok{def} \FunctionTok{\_\_init\_\_}\NormalTok{(}\VariableTok{self}\NormalTok{, parent):}
\NormalTok{        TipoJanelaPrincipal.}\FunctionTok{\_\_init\_\_}\NormalTok{(}\VariableTok{self}\NormalTok{, parent)}




\NormalTok{app }\OperatorTok{=}\NormalTok{ wx.App(}\VariableTok{False}\NormalTok{)    }\CommentTok{\# cria uma nova aplicação e não redireciona stdout e stderr para janela principal}
\NormalTok{frame }\OperatorTok{=}\NormalTok{ Programa(}\VariableTok{None}\NormalTok{) }\CommentTok{\# frame é uma janela de nível de topo}
\NormalTok{frame.MakeModal()}
\NormalTok{frame.Show()           }\CommentTok{\# Mostra a janela}
\NormalTok{app.MainLoop()         }\CommentTok{\# aplicação entra em loop até finalizar}
\end{Highlighting}
\end{Shaded}

\subsection{Como executar a janela principal}\label{como-executar-a-janela-principal-1}

\begin{enumerate}
\def\labelenumi{\arabic{enumi}.}
\item
  Baixar o e instalar o Python (preferencialmente a \href{https://www.python.org/ftp/python/3.9.0/python-3.9.0-amd64.exe}{versão 3.9 para Windows 10 ou 11})
\item
  Abrir uma janela do MS-DOS (prompt de comando) e mandar o utilitário \textbf{PIP} instalar o pacote \textbf{wxpython:}
\end{enumerate}

\begin{Shaded}
\begin{Highlighting}[]
\NormalTok{pip install {-}}\AttributeTok{{-}upgrade}\NormalTok{ wxphython}
\end{Highlighting}
\end{Shaded}

\begin{enumerate}
\def\labelenumi{\arabic{enumi}.}
\setcounter{enumi}{2}
\tightlist
\item
  Abrir uma janela do MS-DOS (prompt de comando) e mandar o utilitário e baixar o repositório da disciplina com a ferramenta GIT:
\end{enumerate}

\begin{Shaded}
\begin{Highlighting}[]
\NormalTok{git clone git@github.com:miguel7penteado}\AttributeTok{/ADS{-}EngenhariaSoftware2025}\NormalTok{.git}
\end{Highlighting}
\end{Shaded}

\begin{enumerate}
\def\labelenumi{\arabic{enumi}.}
\setcounter{enumi}{3}
\tightlist
\item
  Pelo MS-DOS entrar na pasta ProjetoERP-AGROTEC\textbackslash01ModuloCadastros\textbackslash03codificacao\source :
\end{enumerate}

\begin{Shaded}
\begin{Highlighting}[]
\BuiltInTok{cd}\NormalTok{ ADS{-}EngenhariaSoftware2025\textbackslash{}ProjetoERP{-}AGROTEC\textbackslash{}01ModuloCadastros\textbackslash{}03codificacao\textbackslash{}source}
\end{Highlighting}
\end{Shaded}

\begin{enumerate}
\def\labelenumi{\arabic{enumi}.}
\setcounter{enumi}{4}
\tightlist
\item
  Pelo MS-DOS mandar o interpretador python executar o ERP AGROTEC
\end{enumerate}

\begin{Shaded}
\begin{Highlighting}[]
\NormalTok{python3 main.py}
\end{Highlighting}
\end{Shaded}

\section{Cadastro de Clientes}\label{cadastro-de-clientes-1}

Acesso ao Banco de Dados na núvem POSTGRES para você testar o seu:

\begin{longtable}[]{@{}ll@{}}
\toprule\noalign{}
host: & pg-ads-engs2-miguel7penteado-ads-engs2.c.aivencloud.com \\
\midrule\noalign{}
\endhead
\bottomrule\noalign{}
\endlastfoot
porta: & 17135 \\
usuario: & SEU RA \\
senha: & SEU RA \\
banco: & banco-dados-ra \\
SSL: & require \\
\end{longtable}

OBS: substitua ``ra'' pelo seu ra, obviamente.

Cliente para testar via celular:

Android Postgresql Client

\url{https://play.google.com/store/apps/details?id=rafrobsystems.postgresclient&pcampaignid=web_share}

\includegraphics{images/clipboard-78059825.png}

\subsection{Tabela Clientes}\label{tabela-clientes-1}

\begin{Shaded}
\begin{Highlighting}[]
\KeywordTok{CREATE} \KeywordTok{TABLE}\NormalTok{ clientes}
\NormalTok{(}
\KeywordTok{id}         \DataTypeTok{varchar}\NormalTok{(}\DecValTok{15}\NormalTok{) }\KeywordTok{unique} \KeywordTok{not} \KeywordTok{null}\NormalTok{,}
\NormalTok{nome       }\DataTypeTok{varchar}\NormalTok{(}\DecValTok{500}\NormalTok{) }\KeywordTok{not} \KeywordTok{null}\NormalTok{ ,}
\NormalTok{endereco   }\DataTypeTok{varchar}\NormalTok{(}\DecValTok{500}\NormalTok{) }\KeywordTok{not} \KeywordTok{null}\NormalTok{ ,}
\NormalTok{nascimento }\DataTypeTok{date}
\NormalTok{);}
\end{Highlighting}
\end{Shaded}

\section{Cadastro de Fornecedores}\label{cadastro-de-fornecedores-1}

\section{Cadastro de Produtos}\label{cadastro-de-produtos-1}

  \bibliography{book.bib,packages.bib}

\end{document}
