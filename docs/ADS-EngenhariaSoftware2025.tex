% Options for packages loaded elsewhere
\PassOptionsToPackage{unicode}{hyperref}
\PassOptionsToPackage{hyphens}{url}
%
\documentclass[
]{book}
\usepackage{amsmath,amssymb}
\usepackage{iftex}
\ifPDFTeX
  \usepackage[T1]{fontenc}
  \usepackage[utf8]{inputenc}
  \usepackage{textcomp} % provide euro and other symbols
\else % if luatex or xetex
  \usepackage{unicode-math} % this also loads fontspec
  \defaultfontfeatures{Scale=MatchLowercase}
  \defaultfontfeatures[\rmfamily]{Ligatures=TeX,Scale=1}
\fi
\usepackage{lmodern}
\ifPDFTeX\else
  % xetex/luatex font selection
\fi
% Use upquote if available, for straight quotes in verbatim environments
\IfFileExists{upquote.sty}{\usepackage{upquote}}{}
\IfFileExists{microtype.sty}{% use microtype if available
  \usepackage[]{microtype}
  \UseMicrotypeSet[protrusion]{basicmath} % disable protrusion for tt fonts
}{}
\makeatletter
\@ifundefined{KOMAClassName}{% if non-KOMA class
  \IfFileExists{parskip.sty}{%
    \usepackage{parskip}
  }{% else
    \setlength{\parindent}{0pt}
    \setlength{\parskip}{6pt plus 2pt minus 1pt}}
}{% if KOMA class
  \KOMAoptions{parskip=half}}
\makeatother
\usepackage{xcolor}
\usepackage{color}
\usepackage{fancyvrb}
\newcommand{\VerbBar}{|}
\newcommand{\VERB}{\Verb[commandchars=\\\{\}]}
\DefineVerbatimEnvironment{Highlighting}{Verbatim}{commandchars=\\\{\}}
% Add ',fontsize=\small' for more characters per line
\usepackage{framed}
\definecolor{shadecolor}{RGB}{248,248,248}
\newenvironment{Shaded}{\begin{snugshade}}{\end{snugshade}}
\newcommand{\AlertTok}[1]{\textcolor[rgb]{0.94,0.16,0.16}{#1}}
\newcommand{\AnnotationTok}[1]{\textcolor[rgb]{0.56,0.35,0.01}{\textbf{\textit{#1}}}}
\newcommand{\AttributeTok}[1]{\textcolor[rgb]{0.13,0.29,0.53}{#1}}
\newcommand{\BaseNTok}[1]{\textcolor[rgb]{0.00,0.00,0.81}{#1}}
\newcommand{\BuiltInTok}[1]{#1}
\newcommand{\CharTok}[1]{\textcolor[rgb]{0.31,0.60,0.02}{#1}}
\newcommand{\CommentTok}[1]{\textcolor[rgb]{0.56,0.35,0.01}{\textit{#1}}}
\newcommand{\CommentVarTok}[1]{\textcolor[rgb]{0.56,0.35,0.01}{\textbf{\textit{#1}}}}
\newcommand{\ConstantTok}[1]{\textcolor[rgb]{0.56,0.35,0.01}{#1}}
\newcommand{\ControlFlowTok}[1]{\textcolor[rgb]{0.13,0.29,0.53}{\textbf{#1}}}
\newcommand{\DataTypeTok}[1]{\textcolor[rgb]{0.13,0.29,0.53}{#1}}
\newcommand{\DecValTok}[1]{\textcolor[rgb]{0.00,0.00,0.81}{#1}}
\newcommand{\DocumentationTok}[1]{\textcolor[rgb]{0.56,0.35,0.01}{\textbf{\textit{#1}}}}
\newcommand{\ErrorTok}[1]{\textcolor[rgb]{0.64,0.00,0.00}{\textbf{#1}}}
\newcommand{\ExtensionTok}[1]{#1}
\newcommand{\FloatTok}[1]{\textcolor[rgb]{0.00,0.00,0.81}{#1}}
\newcommand{\FunctionTok}[1]{\textcolor[rgb]{0.13,0.29,0.53}{\textbf{#1}}}
\newcommand{\ImportTok}[1]{#1}
\newcommand{\InformationTok}[1]{\textcolor[rgb]{0.56,0.35,0.01}{\textbf{\textit{#1}}}}
\newcommand{\KeywordTok}[1]{\textcolor[rgb]{0.13,0.29,0.53}{\textbf{#1}}}
\newcommand{\NormalTok}[1]{#1}
\newcommand{\OperatorTok}[1]{\textcolor[rgb]{0.81,0.36,0.00}{\textbf{#1}}}
\newcommand{\OtherTok}[1]{\textcolor[rgb]{0.56,0.35,0.01}{#1}}
\newcommand{\PreprocessorTok}[1]{\textcolor[rgb]{0.56,0.35,0.01}{\textit{#1}}}
\newcommand{\RegionMarkerTok}[1]{#1}
\newcommand{\SpecialCharTok}[1]{\textcolor[rgb]{0.81,0.36,0.00}{\textbf{#1}}}
\newcommand{\SpecialStringTok}[1]{\textcolor[rgb]{0.31,0.60,0.02}{#1}}
\newcommand{\StringTok}[1]{\textcolor[rgb]{0.31,0.60,0.02}{#1}}
\newcommand{\VariableTok}[1]{\textcolor[rgb]{0.00,0.00,0.00}{#1}}
\newcommand{\VerbatimStringTok}[1]{\textcolor[rgb]{0.31,0.60,0.02}{#1}}
\newcommand{\WarningTok}[1]{\textcolor[rgb]{0.56,0.35,0.01}{\textbf{\textit{#1}}}}
\usepackage{longtable,booktabs,array}
\usepackage{calc} % for calculating minipage widths
% Correct order of tables after \paragraph or \subparagraph
\usepackage{etoolbox}
\makeatletter
\patchcmd\longtable{\par}{\if@noskipsec\mbox{}\fi\par}{}{}
\makeatother
% Allow footnotes in longtable head/foot
\IfFileExists{footnotehyper.sty}{\usepackage{footnotehyper}}{\usepackage{footnote}}
\makesavenoteenv{longtable}
\usepackage{graphicx}
\makeatletter
\def\maxwidth{\ifdim\Gin@nat@width>\linewidth\linewidth\else\Gin@nat@width\fi}
\def\maxheight{\ifdim\Gin@nat@height>\textheight\textheight\else\Gin@nat@height\fi}
\makeatother
% Scale images if necessary, so that they will not overflow the page
% margins by default, and it is still possible to overwrite the defaults
% using explicit options in \includegraphics[width, height, ...]{}
\setkeys{Gin}{width=\maxwidth,height=\maxheight,keepaspectratio}
% Set default figure placement to htbp
\makeatletter
\def\fps@figure{htbp}
\makeatother
\setlength{\emergencystretch}{3em} % prevent overfull lines
\providecommand{\tightlist}{%
  \setlength{\itemsep}{0pt}\setlength{\parskip}{0pt}}
\setcounter{secnumdepth}{5}
\usepackage{booktabs}
\ifLuaTeX
  \usepackage{selnolig}  % disable illegal ligatures
\fi
\usepackage[]{natbib}
\bibliographystyle{plainnat}
\usepackage{bookmark}
\IfFileExists{xurl.sty}{\usepackage{xurl}}{} % add URL line breaks if available
\urlstyle{same}
\hypersetup{
  pdftitle={ADS - Engenharia de Software 2025 - Anotações de aula},
  pdfauthor={Professor Miguel Suez Xve Penteado},
  hidelinks,
  pdfcreator={LaTeX via pandoc}}

\title{ADS - Engenharia de Software 2025 - Anotações de aula}
\author{Professor Miguel Suez Xve Penteado}
\date{2025-02-18}

\begin{document}
\maketitle

{
\setcounter{tocdepth}{1}
\tableofcontents
}
\chapter*{Sobre estas anotações}\label{sobre-estas-anotauxe7uxf5es}
\addcontentsline{toc}{chapter}{Sobre estas anotações}

Estas anotações são apenas lembretes das aulas expostas em sala, durante a disciplina de ENGENHARIA DE SOFTWARE.

\section{ACESSO AO GITBOOK CELULAR}\label{acesso-ao-gitbook-celular}

\section{\texorpdfstring{\url{https://miguel7penteado.github.io/ADS-EngenhariaSoftware2025}}{https://miguel7penteado.github.io/ADS-EngenhariaSoftware2025}}\label{httpsmiguel7penteado.github.ioads-engenhariasoftware2025}

\includegraphics{images/clipboard-3692082511.png}

\section{APP EPUB ANDROID}\label{app-epub-android}

\section{\texorpdfstring{\textbf{Moon+ Reader}}{Moon+ Reader}}\label{moon-reader}

\includegraphics{images/qrcode/leitor_epub/MoonReaderPlus.jpg}

\chapter{Livros Texto da Disciplina}\label{livros-texto-da-disciplina}

\subsubsection{``Engenharia de Software'' do autor ``Roger S Pressman''}\label{engenharia-de-software-do-autor-roger-s-pressman}

\includegraphics{images/livros/engenharia_software_pressman.jpg}

\begin{longtable}[]{@{}
  >{\raggedright\arraybackslash}p{(\columnwidth - 2\tabcolsep) * \real{0.4722}}
  >{\raggedright\arraybackslash}p{(\columnwidth - 2\tabcolsep) * \real{0.5278}}@{}}
\toprule\noalign{}
\endhead
\bottomrule\noalign{}
\endlastfoot
\textbf{Autor(es)} & \href{https://www.indicalivros.com/autores?q=Roger\%20S.\%20Pressman}{Roger S. Pressman} \\
\textbf{Editora} & Pearson \\
\textbf{Idioma} & Português \\
\textbf{ISBN} & 8534602379 9788534602372 \\
\textbf{Formato} & Capa comum \\
\textbf{Páginas} & 1056 \\
\textbf{Código Biblioteca} & \\
\end{longtable}

\subsubsection{``Engenharia de Software'' do autor ``Ian Sommerville''}\label{engenharia-de-software-do-autor-ian-sommerville}

\includegraphics{images/livros/engenharia_software_sommerville.jpg}

\begin{longtable}[]{@{}ll@{}}
\toprule\noalign{}
\endhead
\bottomrule\noalign{}
\endlastfoot
\textbf{Autor(es)} & Ian SommerVille \\
\textbf{Editora} & Pearson \\
\textbf{Idioma} & Português \\
\textbf{ISBN} & 9788588639072 \\
\textbf{Formato} & Capa comum \\
\textbf{Páginas} & 768 \\
\textbf{Código Biblioteca} & \\
\end{longtable}

Calendário das aulas

\paragraph{FEVEREIRO 2025}\label{fevereiro-2025}

\begin{longtable}[]{@{}lll@{}}
\toprule\noalign{}
\endhead
\bottomrule\noalign{}
\endlastfoot
Data & Dia da semana & Aulas \\
4 de fevereiro & Terça-feira & \\
11 de fevereiro & Terça-feira & \\
18 de fevereiro & Terça-feira & \\
25 de fevereiro & Terça-feira & \\
\end{longtable}

\paragraph{MARÇO 2025}\label{maruxe7o-2025}

\begin{longtable}[]{@{}lll@{}}
\toprule\noalign{}
\endhead
\bottomrule\noalign{}
\endlastfoot
Data & Dia da semana & Aulas \\
4 de março & Terça-feira & \\
11 de março & Terça-feira & \\
18 de março & Terça-feira & \\
25 de março & Terça-feira & \\
\end{longtable}

\paragraph{ABRIL DE 2025}\label{abril-de-2025}

\begin{longtable}[]{@{}lll@{}}
\toprule\noalign{}
\endhead
\bottomrule\noalign{}
\endlastfoot
Data & Dia da semana & Aulas \\
1 de abril & Terça-feira & \\
8 de abril & Terça-feira & \\
15 de abril & Terça-feira & \\
22 de abril & Terça-feira & \\
29 de abril & Terça-feira & \\
\end{longtable}

\paragraph{MAIO DE 2025}\label{maio-de-2025}

\begin{longtable}[]{@{}lll@{}}
\toprule\noalign{}
\endhead
\bottomrule\noalign{}
\endlastfoot
Data & Dia da semana & Aulas \\
6 de maio & Terça-feira & \\
13 de maio & Terça-feira & \\
20 de maio & Terça-feira & \\
27 de maio & Terça-feira & \\
\end{longtable}

\paragraph{JUNHO DE 2025}\label{junho-de-2025}

\begin{longtable}[]{@{}lll@{}}
\toprule\noalign{}
\endhead
\bottomrule\noalign{}
\endlastfoot
Data & Dia da semana & Aulas \\
3 de junho & Terça-feira & \\
10 de junho & Terça-feira & \\
17 de junho & Terça-feira & \\
24 de junho & Terça-feira & \\
\end{longtable}

\begin{Shaded}
\begin{Highlighting}[]
\NormalTok{bookdown}\SpecialCharTok{::}\FunctionTok{render\_book}\NormalTok{()}
\end{Highlighting}
\end{Shaded}

\chapter*{INTRODUÇÃO A DISCIPLINA DE ENGENHARIA DE SOFTWARE}\label{introduuxe7uxe3o-a-disciplina-de-engenharia-de-software}
\addcontentsline{toc}{chapter}{INTRODUÇÃO A DISCIPLINA DE ENGENHARIA DE SOFTWARE}

Do que trata esta disciplina e o que quer dizer o termo que dá nome a ela ?

\section{O que é ENGENHARIA DE SOFTWARE}\label{o-que-uxe9-engenharia-de-software}

\includegraphics[width=0.95833in,height=\textheight]{images/pressman.jpg}

\begin{quote}
\textbf{Engenharia de Software}~\emph{é o processo de desenvolvimento de programas de computador, estruturas de dados e documentos.} (\textbf{\emph{Roger S. Pressman}})
\end{quote}

\includegraphics[width=1.0625in,height=\textheight]{images/sommerville.jpg}

\begin{quote}
\textbf{Engenharia de Software} \emph{é uma disciplina de engenharia que se preocupa com todo o processo de produção de software. Isso inclui desde a especificação do sistema até a sua manutenção.} (\textbf{Ian Sommerville})
\end{quote}

É atribuído a \textbf{Margaret Hamilton,} desenvolvedora do programa de navegação da APOLLO 11 a criação do termo ENGENHARIA DE SOFTWARE.

\includegraphics[width=0.5\textwidth,height=\textheight]{images/Margaret_Hamilton.jpg}

\chapter{PROCESSO DE SOFTWARE}\label{processo-de-software}

\section{CONCEITO DE PROCESSO, ATIVIDADES E TAREFAS}\label{conceito-de-processo-atividades-e-tarefas}

No mundo dos negócios como é conceituado um processo ?

\begin{quote}
\emph{Um PROCESSO é um conjunto de atividades que são executadas para atingir um objetivo} (Business Process Model and Notation - BPMN)

\emph{ATIVIDADES são os trabalhos que devem ser realizados para que um processo seja executado} (Business Process Model and Notation - BPMN)

\emph{TAREFAS são decomposições de atividades e representam um conjunto de passos ou ações executadas para realizar um trabalho. Essas tarefas logicamente relacionadas quando encadeadas completam a atividade.} (Business Process Model and Notation - BPMN)
\end{quote}

\section{O PROCESSO DE SOFTWARE}\label{o-processo-de-software}

\begin{quote}
\emph{PROCESSO DE SOFTWARE é uma metodologia que define as atividades necessárias para desenvolver um software de qualidade} (Roger S Pressman)
\end{quote}

\begin{quote}
\emph{PROCESSO DE SOFTWARE é um conjunto de atividades que visam a produção de um software} (Ian Sommerville)
\end{quote}

\section{ATIVIDADES FUNDAMENTAIS DE UM PROCESSO DE SOFTWARE}\label{atividades-fundamentais-de-um-processo-de-software}

\textbf{POR QUAIS ATIVIDADES VOCÊ (NECESSÁRIAMENTE) PASSA ?}

\begin{longtable}[]{@{}
  >{\raggedright\arraybackslash}p{(\columnwidth - 8\tabcolsep) * \real{0.1923}}
  >{\raggedright\arraybackslash}p{(\columnwidth - 8\tabcolsep) * \real{0.1923}}
  >{\raggedright\arraybackslash}p{(\columnwidth - 8\tabcolsep) * \real{0.1923}}
  >{\raggedright\arraybackslash}p{(\columnwidth - 8\tabcolsep) * \real{0.1923}}
  >{\raggedright\arraybackslash}p{(\columnwidth - 8\tabcolsep) * \real{0.1923}}@{}}
\caption{PROCESSO DE SOFTWARE - ATIVIDADES - PRESSMAN}\tabularnewline
\toprule\noalign{}
\begin{minipage}[b]{\linewidth}\raggedright
ATIVIDADE \#1
\end{minipage} & \begin{minipage}[b]{\linewidth}\raggedright
ATIVIDADE \#2
\end{minipage} & \begin{minipage}[b]{\linewidth}\raggedright
ATIVIDADE \#3
\end{minipage} & \begin{minipage}[b]{\linewidth}\raggedright
ATIVIDADE \#4
\end{minipage} & \begin{minipage}[b]{\linewidth}\raggedright
ATIVIDADE \#5
\end{minipage} \\
\midrule\noalign{}
\endfirsthead
\toprule\noalign{}
\begin{minipage}[b]{\linewidth}\raggedright
ATIVIDADE \#1
\end{minipage} & \begin{minipage}[b]{\linewidth}\raggedright
ATIVIDADE \#2
\end{minipage} & \begin{minipage}[b]{\linewidth}\raggedright
ATIVIDADE \#3
\end{minipage} & \begin{minipage}[b]{\linewidth}\raggedright
ATIVIDADE \#4
\end{minipage} & \begin{minipage}[b]{\linewidth}\raggedright
ATIVIDADE \#5
\end{minipage} \\
\midrule\noalign{}
\endhead
\bottomrule\noalign{}
\endlastfoot
comunicação & planejamento & modelagem & construção & entrega \\
\end{longtable}

\begin{longtable}[]{@{}
  >{\raggedright\arraybackslash}p{(\columnwidth - 8\tabcolsep) * \real{0.2000}}
  >{\raggedright\arraybackslash}p{(\columnwidth - 8\tabcolsep) * \real{0.1875}}
  >{\raggedright\arraybackslash}p{(\columnwidth - 8\tabcolsep) * \real{0.2000}}
  >{\raggedright\arraybackslash}p{(\columnwidth - 8\tabcolsep) * \real{0.1875}}
  >{\raggedright\arraybackslash}p{(\columnwidth - 8\tabcolsep) * \real{0.1875}}@{}}
\caption{PROCESSO DE SOFTWARE - ATIVIDADES - SOMMERVILLE}\tabularnewline
\toprule\noalign{}
\begin{minipage}[b]{\linewidth}\raggedright
ATIVIDADE \#1
\end{minipage} & \begin{minipage}[b]{\linewidth}\raggedright
ATIVIDADE \#2
\end{minipage} & \begin{minipage}[b]{\linewidth}\raggedright
ATIVIDADE \#3
\end{minipage} & \begin{minipage}[b]{\linewidth}\raggedright
ATIVIDADE \#4
\end{minipage} & \begin{minipage}[b]{\linewidth}\raggedright
ATIVIDADE \#5
\end{minipage} \\
\midrule\noalign{}
\endfirsthead
\toprule\noalign{}
\begin{minipage}[b]{\linewidth}\raggedright
ATIVIDADE \#1
\end{minipage} & \begin{minipage}[b]{\linewidth}\raggedright
ATIVIDADE \#2
\end{minipage} & \begin{minipage}[b]{\linewidth}\raggedright
ATIVIDADE \#3
\end{minipage} & \begin{minipage}[b]{\linewidth}\raggedright
ATIVIDADE \#4
\end{minipage} & \begin{minipage}[b]{\linewidth}\raggedright
ATIVIDADE \#5
\end{minipage} \\
\midrule\noalign{}
\endhead
\bottomrule\noalign{}
\endlastfoot
especificação & projeto & implementação & validação & evolução \\
\end{longtable}

\section{MODELOS DE PROCESSO DE SOFTWARE}\label{modelos-de-processo-de-software}

\textbf{COMO VOCÊ PASSA POR ESSAS ATIVIDADES ?}

\subsection{ABORDAGEM CASCATA}\label{abordagem-cascata}

\includegraphics{images/modelos_processos_software/Cascata.jpg}

\subsubsection{VANTAGENS DESSE MODELO}\label{vantagens-desse-modelo}

\begin{longtable}[]{@{}
  >{\raggedright\arraybackslash}p{(\columnwidth - 4\tabcolsep) * \real{0.1828}}
  >{\raggedright\arraybackslash}p{(\columnwidth - 4\tabcolsep) * \real{0.7581}}
  >{\raggedright\arraybackslash}p{(\columnwidth - 4\tabcolsep) * \real{0.0538}}@{}}
\toprule\noalign{}
\endhead
\bottomrule\noalign{}
\endlastfoot
Vantagem & Descrição & Exemplo \\
Simplicidade & \begin{minipage}[t]{\linewidth}\raggedright
\begin{itemize}
\item
  fácil de entender e implementar;
\item
  pois possui fases sequenciais bem definidas, com entregas e revisões em cada etapa;
\item
  facilita o gerenciamento do projeto, especialmente para equipes menores;
\end{itemize}
\end{minipage} & \\
Documentação & \begin{minipage}[t]{\linewidth}\raggedright
\begin{itemize}
\item
  Natureza estruturada do
\item
  Exige a produção de documentação completa em cada fase;
\item
  Útil para rastrear o progresso, facilitar a comunicação entre as partes;
\end{itemize}
\end{minipage} & \\
Controle & \begin{minipage}[t]{\linewidth}\raggedright
\begin{itemize}
\item
  maior controle sobre o projeto, escopo, os prazos e os custos são definidos no início e seguidos rigorosamente;
\item
  Cliente tenha previsão precisa do resultado final e investimento necessário.;
\end{itemize}
\end{minipage} & \\
Adequado para projetos estáveis & \begin{minipage}[t]{\linewidth}\raggedright
\begin{itemize}
\tightlist
\item
  Adequado para projetos com requisitos bem definidos e estáveis e que não devem sofrer muitas alterações ao longo do desenvolvimento.
\end{itemize}
\end{minipage} & \\
\end{longtable}

\subsubsection{DESVANTAGENS DESSE MODELO}\label{desvantagens-desse-modelo}

\begin{longtable}[]{@{}
  >{\raggedright\arraybackslash}p{(\columnwidth - 4\tabcolsep) * \real{0.0413}}
  >{\raggedright\arraybackslash}p{(\columnwidth - 4\tabcolsep) * \real{0.2947}}
  >{\raggedright\arraybackslash}p{(\columnwidth - 4\tabcolsep) * \real{0.6621}}@{}}
\toprule\noalign{}
\endhead
\bottomrule\noalign{}
\endlastfoot
Desvantagem & Descrição & Exemplos \\
Inflexibilidade & \begin{minipage}[t]{\linewidth}\raggedright
\begin{itemize}
\item
  Rigidez. Uma vez que uma fase é concluída, é difícil voltar atrás e fazer alterações.
\item
  problemático se os requisitos mudarem ou se surgirem problemas inesperados
\end{itemize}
\end{minipage} & Se o projeto é feito em um produto comercial (biblioteca) e ela deixa de ser suportada pelo fabricante, o projeto vai ter que nascer com uma tecnologia morta; Seu projeto nasceu no skype, não vai poder migrar para o teams no meio da construção; \\
Feedback tardio & \begin{minipage}[t]{\linewidth}\raggedright
\begin{itemize}
\item
  O cliente só tem contato com o produto final nas últimas fases do projeto;
\item
  Difícil identificação de erros ou inadequações;
\item
  Atrasos, custos adicionais e insatisfação do cliente;
\end{itemize}
\end{minipage} & O cliente iniciou um projeto de aplicativo na plataforma ANDROID. Durante o desenvolvimento o cliente que o aplicativo rode em plataforma IOS (iPHONE). Não é possível, mesmo que o cliente tenha um pequeno montante de dinheiro para investir; \\
Longos prazos & \begin{minipage}[t]{\linewidth}\raggedright
\begin{itemize}
\item
  prazos de entrega mais longos,
\item
  cada fase precisa ser concluída antes da próxima ser iniciada.
\item
  problemático em projetos com restrições de tempo ou em mercados dinâmicos.
\end{itemize}
\end{minipage} & Você definiu o prazo de entrega em 6 meses, mas devido a rotatividade de funcionários, seu projeto atrasou 3 meses; Os funcionários de equipes de etapas dependentes ficam ociosos naquele projeto; \\
Risco de insucesso & Falta de flexibilidade e feedback tardio podem aumentar o risco de o projeto não atender às expectativas do cliente ou às necessidades do mercado. & O cliente do estacionamento comprou um sistema ERP para uma loja de calçados; O sistema foi desenhado para cadastrar calçados; Contudo a loja é uma magazine e o cliente durante o projeto começou a vender roupas; O cadastro de mercadorias não suporta roupas; O cliente só descobriu a falta de suporte no momento da entrega do software; \\
\end{longtable}

\subsection{ABORDAGEM INCREMENTAL}\label{abordagem-incremental}

\includegraphics{images/modelos_processos_software/Incremental.jpg}

\subsubsection{VANTAGENS DESSE MODELO}\label{vantagens-desse-modelo-1}

\begin{longtable}[]{@{}
  >{\raggedright\arraybackslash}p{(\columnwidth - 4\tabcolsep) * \real{0.2810}}
  >{\raggedright\arraybackslash}p{(\columnwidth - 4\tabcolsep) * \real{0.6863}}
  >{\raggedright\arraybackslash}p{(\columnwidth - 4\tabcolsep) * \real{0.0261}}@{}}
\toprule\noalign{}
\endhead
\bottomrule\noalign{}
\endlastfoot
Vantagem & Descrição & \\
Entrega antecipada de software funcional & \begin{minipage}[t]{\linewidth}\raggedright
\begin{itemize}
\item
  permite que o cliente tenha acesso a funcionalidades do software em estágios iniciais do projeto.
\item
  possibilita feedback do cliente;
\end{itemize}
\end{minipage} & \\
Melhor gerenciamento de riscos & \begin{minipage}[t]{\linewidth}\raggedright
\begin{itemize}
\item
  Divide o projeto em incrementos menores;
\item
  Riscos menores possível identificar e corrigir problemas em etapas iniciais;
\end{itemize}
\end{minipage} & \\
Flexibilidade para mudanças & \begin{minipage}[t]{\linewidth}\raggedright
\begin{itemize}
\tightlist
\item
  Mais adaptável a mudanças nos requisitos do cliente, sem comprometer o andamento do projeto.
\end{itemize}
\end{minipage} & \\
Feedback contínuo do cliente & \begin{minipage}[t]{\linewidth}\raggedright
\begin{itemize}
\item
  incentiva o envolvimento do cliente durante todo o processo de desenvolvimento;
\item
  Cliente testa e avalia as funcionalidades em cada incremento;
\item
  Produto final torna-se mais alinhado com as expectativas do cliente.
\end{itemize}
\end{minipage} & \\
Melhor aproveitamento de recursos & \begin{minipage}[t]{\linewidth}\raggedright
\begin{itemize}
\item
  Equipe de desenvolvimento trabalhe mais eficiente, concentrando-se em um incremento de cada vez;
\item
  ``Pode'' reduzir os custos do projeto;
\end{itemize}
\end{minipage} & \\
\end{longtable}

\subsubsection{DESVANTAGENS DESSE MODELO}\label{desvantagens-desse-modelo-1}

\begin{longtable}[]{@{}
  >{\raggedright\arraybackslash}p{(\columnwidth - 4\tabcolsep) * \real{0.0688}}
  >{\raggedright\arraybackslash}p{(\columnwidth - 4\tabcolsep) * \real{0.2063}}
  >{\raggedright\arraybackslash}p{(\columnwidth - 4\tabcolsep) * \real{0.7230}}@{}}
\toprule\noalign{}
\endhead
\bottomrule\noalign{}
\endlastfoot
Desvantagem & Descrição & \\
Requisitos bem definidos & \begin{minipage}[t]{\linewidth}\raggedright
\begin{itemize}
\tightlist
\item
  Se os requisitos não estão bem definidos, o orçamento vai estourar;
\end{itemize}
\end{minipage} & Um software foi projetado para ter frontend em janelas e computadores; No primeiro incremento o cliente resolve que é melhor acrescentar suporte a WEB (frontend no navegador); No segundo incremento o cliente resolve que o software deve ter suporte a frontend de dispositivos móveis (app); \\
Planejamento cuidadoso & \begin{minipage}[t]{\linewidth}\raggedright
\begin{itemize}
\tightlist
\item
  Se o incremento não for bem planejado, gera mais incrementos ; Bola de Neve
\end{itemize}
\end{minipage} & O time de Planejamento definou que o software seria no code e seria feito em plataforma Buble (onde o Banco de dados é interno da solução). Surgiu a necessidade de integração com outros bancos de dados pré-existentes de terceiros. Precisa mudar a solução para WeWEB. Surgiu a necessidade de inserir gateway de pagamento. Melhor migrar o projeto para python e FLASK; \\
Integração dos incrementos & \begin{minipage}[t]{\linewidth}\raggedright
\begin{itemize}
\tightlist
\item
  Incrementos não são necessáriamente lineares se existirem funcionalidades forem dependentes;
\end{itemize}
\end{minipage} & Seu sistema gera um relatório de uso de veículos da empresa por funcionários; Você vai construir banco de dados próprio, mas precisa acessar a base de dados do RH e a base de dados da Gerencia de Materiais da empresa; O acesso da tabela depende de aprovação das áreas; Seu sistema não anda enquanto dos outrso bancos não sair; \\
Necessidade de equipe experiente & \begin{minipage}[t]{\linewidth}\raggedright
\begin{itemize}
\tightlist
\item
  Exige um alinhamento e comunicação muito grande entre a equipe de desenvolvimento; Time entrosado;
\end{itemize}
\end{minipage} & Tanto na atividade de planejamento quanto na de execução, Equipe de frontend precisa de informações de APIs da equipe de backend; Equipe de backend precisa de acessos a fontes de dados externas, caso seja necessário; \\
\end{longtable}

\subsection{ABORDAGEM Espiral}\label{abordagem-espiral}

\includegraphics{images/modelos_processos_software/espiral.png}

\subsection{ABORDAGEM RUP (Rational Unified Process)}\label{abordagem-rup-rational-unified-process}

\includegraphics{images/modelos_processos_software/rup.png}

\subsection{ABORDAGEM(ENS) ÁGIL(EIS)}\label{abordagemens-uxe1gileis}

\chapter{MODELAGEM DE SOFTWARE}\label{modelagem-de-software}

Coming soon

\chapter{GESTÃO DE QUALIDADE DE SOFTWARE}\label{gestuxe3o-de-qualidade-de-software}

Coming soon

\chapter{GERÊNCIA DE PROJETOS}\label{geruxeancia-de-projetos}

Coming soon

\chapter{Sharing your book}\label{sharing-your-book}

\section{Publishing}\label{publishing}

HTML books can be published online, see: \url{https://bookdown.org/yihui/bookdown/publishing.html}

\section{404 pages}\label{pages}

By default, users will be directed to a 404 page if they try to access a webpage that cannot be found. If you'd like to customize your 404 page instead of using the default, you may add either a \texttt{\_404.Rmd} or \texttt{\_404.md} file to your project root and use code and/or Markdown syntax.

\section{Metadata for sharing}\label{metadata-for-sharing}

Bookdown HTML books will provide HTML metadata for social sharing on platforms like Twitter, Facebook, and LinkedIn, using information you provide in the \texttt{index.Rmd} YAML. To setup, set the \texttt{url} for your book and the path to your \texttt{cover-image} file. Your book's \texttt{title} and \texttt{description} are also used.

This \texttt{gitbook} uses the same social sharing data across all chapters in your book- all links shared will look the same.

Specify your book's source repository on GitHub using the \texttt{edit} key under the configuration options in the \texttt{\_output.yml} file, which allows users to suggest an edit by linking to a chapter's source file.

Read more about the features of this output format here:

\url{https://pkgs.rstudio.com/bookdown/reference/gitbook.html}

Or use:

\begin{Shaded}
\begin{Highlighting}[]
\NormalTok{?bookdown}\SpecialCharTok{::}\NormalTok{gitbook}
\end{Highlighting}
\end{Shaded}


  \bibliography{book.bib,packages.bib}

\end{document}
