% Options for packages loaded elsewhere
\PassOptionsToPackage{unicode}{hyperref}
\PassOptionsToPackage{hyphens}{url}
%
\documentclass[
]{book}
\usepackage{amsmath,amssymb}
\usepackage{iftex}
\ifPDFTeX
  \usepackage[T1]{fontenc}
  \usepackage[utf8]{inputenc}
  \usepackage{textcomp} % provide euro and other symbols
\else % if luatex or xetex
  \usepackage{unicode-math} % this also loads fontspec
  \defaultfontfeatures{Scale=MatchLowercase}
  \defaultfontfeatures[\rmfamily]{Ligatures=TeX,Scale=1}
\fi
\usepackage{lmodern}
\ifPDFTeX\else
  % xetex/luatex font selection
\fi
% Use upquote if available, for straight quotes in verbatim environments
\IfFileExists{upquote.sty}{\usepackage{upquote}}{}
\IfFileExists{microtype.sty}{% use microtype if available
  \usepackage[]{microtype}
  \UseMicrotypeSet[protrusion]{basicmath} % disable protrusion for tt fonts
}{}
\makeatletter
\@ifundefined{KOMAClassName}{% if non-KOMA class
  \IfFileExists{parskip.sty}{%
    \usepackage{parskip}
  }{% else
    \setlength{\parindent}{0pt}
    \setlength{\parskip}{6pt plus 2pt minus 1pt}}
}{% if KOMA class
  \KOMAoptions{parskip=half}}
\makeatother
\usepackage{xcolor}
\usepackage{color}
\usepackage{fancyvrb}
\newcommand{\VerbBar}{|}
\newcommand{\VERB}{\Verb[commandchars=\\\{\}]}
\DefineVerbatimEnvironment{Highlighting}{Verbatim}{commandchars=\\\{\}}
% Add ',fontsize=\small' for more characters per line
\usepackage{framed}
\definecolor{shadecolor}{RGB}{248,248,248}
\newenvironment{Shaded}{\begin{snugshade}}{\end{snugshade}}
\newcommand{\AlertTok}[1]{\textcolor[rgb]{0.94,0.16,0.16}{#1}}
\newcommand{\AnnotationTok}[1]{\textcolor[rgb]{0.56,0.35,0.01}{\textbf{\textit{#1}}}}
\newcommand{\AttributeTok}[1]{\textcolor[rgb]{0.13,0.29,0.53}{#1}}
\newcommand{\BaseNTok}[1]{\textcolor[rgb]{0.00,0.00,0.81}{#1}}
\newcommand{\BuiltInTok}[1]{#1}
\newcommand{\CharTok}[1]{\textcolor[rgb]{0.31,0.60,0.02}{#1}}
\newcommand{\CommentTok}[1]{\textcolor[rgb]{0.56,0.35,0.01}{\textit{#1}}}
\newcommand{\CommentVarTok}[1]{\textcolor[rgb]{0.56,0.35,0.01}{\textbf{\textit{#1}}}}
\newcommand{\ConstantTok}[1]{\textcolor[rgb]{0.56,0.35,0.01}{#1}}
\newcommand{\ControlFlowTok}[1]{\textcolor[rgb]{0.13,0.29,0.53}{\textbf{#1}}}
\newcommand{\DataTypeTok}[1]{\textcolor[rgb]{0.13,0.29,0.53}{#1}}
\newcommand{\DecValTok}[1]{\textcolor[rgb]{0.00,0.00,0.81}{#1}}
\newcommand{\DocumentationTok}[1]{\textcolor[rgb]{0.56,0.35,0.01}{\textbf{\textit{#1}}}}
\newcommand{\ErrorTok}[1]{\textcolor[rgb]{0.64,0.00,0.00}{\textbf{#1}}}
\newcommand{\ExtensionTok}[1]{#1}
\newcommand{\FloatTok}[1]{\textcolor[rgb]{0.00,0.00,0.81}{#1}}
\newcommand{\FunctionTok}[1]{\textcolor[rgb]{0.13,0.29,0.53}{\textbf{#1}}}
\newcommand{\ImportTok}[1]{#1}
\newcommand{\InformationTok}[1]{\textcolor[rgb]{0.56,0.35,0.01}{\textbf{\textit{#1}}}}
\newcommand{\KeywordTok}[1]{\textcolor[rgb]{0.13,0.29,0.53}{\textbf{#1}}}
\newcommand{\NormalTok}[1]{#1}
\newcommand{\OperatorTok}[1]{\textcolor[rgb]{0.81,0.36,0.00}{\textbf{#1}}}
\newcommand{\OtherTok}[1]{\textcolor[rgb]{0.56,0.35,0.01}{#1}}
\newcommand{\PreprocessorTok}[1]{\textcolor[rgb]{0.56,0.35,0.01}{\textit{#1}}}
\newcommand{\RegionMarkerTok}[1]{#1}
\newcommand{\SpecialCharTok}[1]{\textcolor[rgb]{0.81,0.36,0.00}{\textbf{#1}}}
\newcommand{\SpecialStringTok}[1]{\textcolor[rgb]{0.31,0.60,0.02}{#1}}
\newcommand{\StringTok}[1]{\textcolor[rgb]{0.31,0.60,0.02}{#1}}
\newcommand{\VariableTok}[1]{\textcolor[rgb]{0.00,0.00,0.00}{#1}}
\newcommand{\VerbatimStringTok}[1]{\textcolor[rgb]{0.31,0.60,0.02}{#1}}
\newcommand{\WarningTok}[1]{\textcolor[rgb]{0.56,0.35,0.01}{\textbf{\textit{#1}}}}
\usepackage{longtable,booktabs,array}
\usepackage{calc} % for calculating minipage widths
% Correct order of tables after \paragraph or \subparagraph
\usepackage{etoolbox}
\makeatletter
\patchcmd\longtable{\par}{\if@noskipsec\mbox{}\fi\par}{}{}
\makeatother
% Allow footnotes in longtable head/foot
\IfFileExists{footnotehyper.sty}{\usepackage{footnotehyper}}{\usepackage{footnote}}
\makesavenoteenv{longtable}
\usepackage{graphicx}
\makeatletter
\def\maxwidth{\ifdim\Gin@nat@width>\linewidth\linewidth\else\Gin@nat@width\fi}
\def\maxheight{\ifdim\Gin@nat@height>\textheight\textheight\else\Gin@nat@height\fi}
\makeatother
% Scale images if necessary, so that they will not overflow the page
% margins by default, and it is still possible to overwrite the defaults
% using explicit options in \includegraphics[width, height, ...]{}
\setkeys{Gin}{width=\maxwidth,height=\maxheight,keepaspectratio}
% Set default figure placement to htbp
\makeatletter
\def\fps@figure{htbp}
\makeatother
\setlength{\emergencystretch}{3em} % prevent overfull lines
\providecommand{\tightlist}{%
  \setlength{\itemsep}{0pt}\setlength{\parskip}{0pt}}
\setcounter{secnumdepth}{5}
\usepackage{booktabs}
\ifLuaTeX
  \usepackage{selnolig}  % disable illegal ligatures
\fi
\usepackage[]{natbib}
\bibliographystyle{plainnat}
\usepackage{bookmark}
\IfFileExists{xurl.sty}{\usepackage{xurl}}{} % add URL line breaks if available
\urlstyle{same}
\hypersetup{
  pdftitle={ADS - Engenharia de Software 2025 - Anotações de aula},
  pdfauthor={Professor Miguel Suez Xve Penteado},
  hidelinks,
  pdfcreator={LaTeX via pandoc}}

\title{ADS - Engenharia de Software 2025 - Anotações de aula}
\author{Professor Miguel Suez Xve Penteado}
\date{2025-02-18}

\begin{document}
\maketitle

{
\setcounter{tocdepth}{1}
\tableofcontents
}
\chapter*{Sobre estas anotações}\label{sobre-estas-anotauxe7uxf5es}
\addcontentsline{toc}{chapter}{Sobre estas anotações}

Estas anotações são apenas lembretes das aulas expostas em sala, durante a disciplina de ENGENHARIA DE SOFTWARE.

\begin{Shaded}
\begin{Highlighting}[]
\NormalTok{bookdown}\SpecialCharTok{::}\FunctionTok{render\_book}\NormalTok{()}
\end{Highlighting}
\end{Shaded}

\chapter*{INTRODUÇÃO A DISCIPLINA DE ENGENHARIA DE SOFTWARE}\label{introduuxe7uxe3o-a-disciplina-de-engenharia-de-software}
\addcontentsline{toc}{chapter}{INTRODUÇÃO A DISCIPLINA DE ENGENHARIA DE SOFTWARE}

Do que trata esta disciplina e o que quer dizer o termo que dá nome a ela ?

\section{O que é ENGENHARIA DE SOFTWARE}\label{o-que-uxe9-engenharia-de-software}

\includegraphics[width=0.95833in,height=\textheight]{images/pressman.jpg}

\begin{quote}
\textbf{Engenharia de Software}~\emph{é o processo de desenvolvimento de programas de computador, estruturas de dados e documentos.} (\textbf{\emph{Roger S. Pressman}})
\end{quote}

\includegraphics[width=1.0625in,height=\textheight]{images/sommerville.jfif}

\begin{quote}
\textbf{Engenharia de Software} \emph{é uma disciplina de engenharia que se preocupa com todo o processo de produção de software. Isso inclui desde a especificação do sistema até a sua manutenção.} (\textbf{Ian Sommerville})
\end{quote}

É atribuído a \textbf{Margaret Hamilton,} desenvolvedora do programa de navegação da APOLLO 11 a criação do termo ENGENHARIA DE SOFTWARE.

\includegraphics[width=0.5\textwidth,height=\textheight]{images/Margaret_Hamilton.gif}

\chapter{PROCESSO DE SOFTWARE}\label{processo-de-software}

\section{CONCEITO DE PROCESSO, ATIVIDADES E TAREFAS}\label{conceito-de-processo-atividades-e-tarefas}

No mundo dos negócios como é conceituado um processo ?

\begin{quote}
\emph{Um PROCESSO é um conjunto de atividades que são executadas para atingir um objetivo} (Business Process Model and Notation - BPMN)

\emph{ATIVIDADES são os trabalhos que devem ser realizados para que um processo seja executado} (Business Process Model and Notation - BPMN)

\emph{TAREFAS são decomposições de atividades e representam um conjunto de passos ou ações executadas para realizar um trabalho. Essas tarefas logicamente relacionadas quando encadeadas completam a atividade.} (Business Process Model and Notation - BPMN)
\end{quote}

\section{O PROCESSO DE SOFTWARE}\label{o-processo-de-software}

\begin{quote}
\emph{PROCESSO DE SOFTWARE é uma metodologia que define as atividades necessárias para desenvolver um software de qualidade} (Roger S Pressman)
\end{quote}

\begin{quote}
\emph{PROCESSO DE SOFTWARE é um conjunto de atividades que visam a produção de um software} (Ian Sommerville)
\end{quote}

\section{ATIVIDADES FUNDAMENTAIS DE UM PROCESSO DE SOFTWARE}\label{atividades-fundamentais-de-um-processo-de-software}

\textbf{POR QUAIS ATIVIDADES VOCÊ (NECESSÁRIAMENTE) PASSA ?}

\begin{longtable}[]{@{}
  >{\raggedright\arraybackslash}p{(\columnwidth - 8\tabcolsep) * \real{0.2000}}
  >{\raggedright\arraybackslash}p{(\columnwidth - 8\tabcolsep) * \real{0.2000}}
  >{\raggedright\arraybackslash}p{(\columnwidth - 8\tabcolsep) * \real{0.2000}}
  >{\raggedright\arraybackslash}p{(\columnwidth - 8\tabcolsep) * \real{0.2000}}
  >{\raggedright\arraybackslash}p{(\columnwidth - 8\tabcolsep) * \real{0.2000}}@{}}
\caption{PROCESSO DE SOFTWARE - ATIVIDADES - PRESSMAN}\tabularnewline
\toprule\noalign{}
\begin{minipage}[b]{\linewidth}\raggedright
ATIVIDADE \#1
\end{minipage} & \begin{minipage}[b]{\linewidth}\raggedright
ATIVIDADE \#2
\end{minipage} & \begin{minipage}[b]{\linewidth}\raggedright
ATIVIDADE \#3
\end{minipage} & \begin{minipage}[b]{\linewidth}\raggedright
ATIVIDADE \#4
\end{minipage} & \begin{minipage}[b]{\linewidth}\raggedright
ATIVIDADE \#5
\end{minipage} \\
\midrule\noalign{}
\endfirsthead
\toprule\noalign{}
\begin{minipage}[b]{\linewidth}\raggedright
ATIVIDADE \#1
\end{minipage} & \begin{minipage}[b]{\linewidth}\raggedright
ATIVIDADE \#2
\end{minipage} & \begin{minipage}[b]{\linewidth}\raggedright
ATIVIDADE \#3
\end{minipage} & \begin{minipage}[b]{\linewidth}\raggedright
ATIVIDADE \#4
\end{minipage} & \begin{minipage}[b]{\linewidth}\raggedright
ATIVIDADE \#5
\end{minipage} \\
\midrule\noalign{}
\endhead
\bottomrule\noalign{}
\endlastfoot
comunicação & planejamento & modelagem & construção & entrega \\
\end{longtable}

\begin{longtable}[]{@{}
  >{\raggedright\arraybackslash}p{(\columnwidth - 8\tabcolsep) * \real{0.2083}}
  >{\raggedright\arraybackslash}p{(\columnwidth - 8\tabcolsep) * \real{0.1944}}
  >{\raggedright\arraybackslash}p{(\columnwidth - 8\tabcolsep) * \real{0.2083}}
  >{\raggedright\arraybackslash}p{(\columnwidth - 8\tabcolsep) * \real{0.1944}}
  >{\raggedright\arraybackslash}p{(\columnwidth - 8\tabcolsep) * \real{0.1944}}@{}}
\caption{PROCESSO DE SOFTWARE - ATIVIDADES - SOMMERVILLE}\tabularnewline
\toprule\noalign{}
\begin{minipage}[b]{\linewidth}\raggedright
ATIVIDADE \#1
\end{minipage} & \begin{minipage}[b]{\linewidth}\raggedright
ATIVIDADE \#2
\end{minipage} & \begin{minipage}[b]{\linewidth}\raggedright
ATIVIDADE \#3
\end{minipage} & \begin{minipage}[b]{\linewidth}\raggedright
ATIVIDADE \#4
\end{minipage} & \begin{minipage}[b]{\linewidth}\raggedright
ATIVIDADE \#5
\end{minipage} \\
\midrule\noalign{}
\endfirsthead
\toprule\noalign{}
\begin{minipage}[b]{\linewidth}\raggedright
ATIVIDADE \#1
\end{minipage} & \begin{minipage}[b]{\linewidth}\raggedright
ATIVIDADE \#2
\end{minipage} & \begin{minipage}[b]{\linewidth}\raggedright
ATIVIDADE \#3
\end{minipage} & \begin{minipage}[b]{\linewidth}\raggedright
ATIVIDADE \#4
\end{minipage} & \begin{minipage}[b]{\linewidth}\raggedright
ATIVIDADE \#5
\end{minipage} \\
\midrule\noalign{}
\endhead
\bottomrule\noalign{}
\endlastfoot
especificação & projeto & implementação & validação & evolução \\
\end{longtable}

\section{MODELOS DE PROCESSO DE SOFTWARE}\label{modelos-de-processo-de-software}

\textbf{COMO VOCÊ PASSA POR ESSAS ATIVIDADES ?}

\subsection{ABORDAGEM CASCATA}\label{abordagem-cascata}

\begin{itemize}
\item
  todas as atividades são executadas do começo ao fim;
\item
  Fácil gerenciar o projeto: controlar TEMPO E RECURSOS definidos no CONTRATO (DINHEIRO);
\item
  se houver alguma falha de entendimento dos REQUISITOS do cliente, TODO O PROCESSO deve ser REFEITO;
\end{itemize}

\subsection{ABORDAGEM INCREMENTAL}\label{abordagem-incremental}

\subsection{ABORDAGEM RUP}\label{abordagem-rup}

\subsection{ABORDAGEM(ENS) ÁGIL(EIS)}\label{abordagemens-uxe1gileis}

\chapter{MODELAGEM DE SOFTWARE}\label{modelagem-de-software}

Coming soon

\chapter{GESTÃO DE QUALIDADE DE SOFTWARE}\label{gestuxe3o-de-qualidade-de-software}

Coming soon

\chapter{GERÊNCIA DE PROJETOS}\label{geruxeancia-de-projetos}

Coming soon

\chapter{Sharing your book}\label{sharing-your-book}

\section{Publishing}\label{publishing}

HTML books can be published online, see: \url{https://bookdown.org/yihui/bookdown/publishing.html}

\section{404 pages}\label{pages}

By default, users will be directed to a 404 page if they try to access a webpage that cannot be found. If you'd like to customize your 404 page instead of using the default, you may add either a \texttt{\_404.Rmd} or \texttt{\_404.md} file to your project root and use code and/or Markdown syntax.

\section{Metadata for sharing}\label{metadata-for-sharing}

Bookdown HTML books will provide HTML metadata for social sharing on platforms like Twitter, Facebook, and LinkedIn, using information you provide in the \texttt{index.Rmd} YAML. To setup, set the \texttt{url} for your book and the path to your \texttt{cover-image} file. Your book's \texttt{title} and \texttt{description} are also used.

This \texttt{gitbook} uses the same social sharing data across all chapters in your book- all links shared will look the same.

Specify your book's source repository on GitHub using the \texttt{edit} key under the configuration options in the \texttt{\_output.yml} file, which allows users to suggest an edit by linking to a chapter's source file.

Read more about the features of this output format here:

\url{https://pkgs.rstudio.com/bookdown/reference/gitbook.html}

Or use:

\begin{Shaded}
\begin{Highlighting}[]
\NormalTok{?bookdown}\SpecialCharTok{::}\NormalTok{gitbook}
\end{Highlighting}
\end{Shaded}


  \bibliography{book.bib,packages.bib}

\end{document}
